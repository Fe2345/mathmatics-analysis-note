\ifx\allfiles\undefined
\documentclass[12pt, a4paper, oneside, UTF8]{ctexbook}
\setCJKmainfont{SimSun}
\def\path{../config}
\usepackage{amsmath}
\usepackage{amsthm}
\usepackage{amssymb}
\usepackage{graphicx}
\usepackage{mathrsfs}
\usepackage{enumitem}
\usepackage{geometry}
\usepackage[colorlinks, linkcolor=black]{hyperref}
\usepackage{stackengine}
\usepackage{yhmath}
\usepackage{extarrows}
\usepackage{unicode-math}
\usepackage{tikz}
\usepackage{tikz-cd}

\usepackage{fancyhdr}
\usepackage[dvipsnames, svgnames]{xcolor}
\usepackage{listings}

\definecolor{mygreen}{rgb}{0,0.6,0}
\definecolor{mygray}{rgb}{0.5,0.5,0.5}
\definecolor{mymauve}{rgb}{0.58,0,0.82}

\graphicspath{ {figure/},{../figure/}, {config/}, {../config/} }

\linespread{1.6}

\geometry{
    top=25.4mm, 
    bottom=25.4mm, 
    left=20mm, 
    right=20mm, 
    headheight=2.17cm, 
    headsep=4mm, 
    footskip=12mm
}

\setenumerate[1]{itemsep=5pt,partopsep=0pt,parsep=\parskip,topsep=5pt}
\setitemize[1]{itemsep=5pt,partopsep=0pt,parsep=\parskip,topsep=5pt}
\setdescription{itemsep=5pt,partopsep=0pt,parsep=\parskip,topsep=5pt}

\lstset{
    language=Mathematica,
    basicstyle=\tt,
    breaklines=true,
    keywordstyle=\bfseries\color{NavyBlue}, 
    emphstyle=\bfseries\color{Rhodamine},
    commentstyle=\itshape\color{black!50!white}, 
    stringstyle=\bfseries\color{PineGreen!90!black},
    columns=flexible,
    numbers=left,
    numberstyle=\footnotesize,
    frame=tb,
    breakatwhitespace=false,
} 
\usepackage[strict]{changepage} 
\usepackage{framed}
\usepackage{tcolorbox}
\tcbuselibrary{most}

\definecolor{greenshade}{rgb}{0.90,1,0.92}
\definecolor{redshade}{rgb}{1.00,0.88,0.88}
\definecolor{brownshade}{rgb}{0.99,0.95,0.9}
\definecolor{lilacshade}{rgb}{0.95,0.93,0.98}
\definecolor{orangeshade}{rgb}{1.00,0.88,0.82}
\definecolor{lightblueshade}{rgb}{0.8,0.92,1}
\definecolor{purple}{rgb}{0.81,0.85,1}

% #### 将 config.tex 中的定理环境的对应部分替换为如下内容
% 定义单独编号,其他四个共用一个编号计数 这里只列举了五种,其他可类似定义(未定义的使用原来的也可)
\newtcbtheorem[number within=section]{defn}%
{定义}{colback=OliveGreen!10,colframe=Green!70,fonttitle=\bfseries}{def}

\newtcbtheorem[number within=section]{lemma}%
{引理}{colback=Salmon!20,colframe=Salmon!90!Black,fonttitle=\bfseries}{lem}

% 使用另一个计数器 use counter from=lemma
\newtcbtheorem[use counter from=lemma, number within=section]{them}%
{定理}{colback=SeaGreen!10!CornflowerBlue!10,colframe=RoyalPurple!55!Aquamarine!100!,fonttitle=\bfseries}{them}

\newtcbtheorem[use counter from=lemma, number within=section]{criterion}%
{准则}{colback=green!5,colframe=green!35!black,fonttitle=\bfseries}{cri}

\newtcbtheorem[use counter from=lemma, number within=section]{corollary}%
{推论}{colback=Emerald!10,colframe=cyan!40!black,fonttitle=\bfseries}{cor}
% colback=red!5,colframe=red!75!black

% 这个颜色我不喜欢
%\newtcbtheorem[number within=section]{proposition}%
%{命题}{colback=red!5,colframe=red!75!black,fonttitle=\bfseries}{cor}

% .... 命题 例 注 证明 解 使用之前的就可以(全文都是这种框框就很丑了),也可以按照上述定义 ...
\renewenvironment{proof}{\par\textbf{证明:}\;}{\qed\par}
\newenvironment{solution}{\par{\textbf{解:}}\;}{\qed\par}
\newtheorem{proposition}{\indent 命题}[section]
\newtheorem{example}{\indent \color{SeaGreen}{例}}[section] % 绿色文字的 例 ,不需要就去除\color{SeaGreen}{}
\newtheorem*{rmk}{\indent 注}
\usepackage{amssymb}
\def\d{\mathrm{d}}
\def\R{\mathbb{R}}
\def\C{\mathbb{C}}
\def\Q{\mathbb{Q}}
\newcommand{\bs}[1]{\boldsymbol{#1}}
\newcommand{\ora}[1]{\overrightarrow{#1}}
\newcommand{\myspace}[1]{\par\vspace{#1\baselineskip}}
\newcommand{\xrowht}[2][0]{\addstackgap[.5\dimexpr#2\relax]{\vphantom{#1}}}
\newenvironment{ca}[1][1]{\linespread{#1} \selectfont \begin{cases}}{\end{cases}}
\newenvironment{vx}[1][1]{\linespread{#1} \selectfont \begin{vmatrix}}{\end{vmatrix}}
\newcommand{\tabincell}[2]{\begin{tabular}{@{}#1@{}}#2\end{tabular}}
\newcommand{\pll}{\kern 0.56em/\kern -0.8em /\kern 0.56em}
\newcommand{\dive}[1][F]{\mathrm{div}\;\bs{#1}}
\newcommand{\rotn}[1][A]{\mathrm{rot}\;\bs{#1}}
\usepackage{xeCJK}
\setCJKmainfont{SimSun}[BoldFont={SimHei}, ItalicFont={KaiTi}] % 设置中文支持

\newcommand{\point}[1]{\item {#1}}
\newenvironment{para}[1]{%
\ifcase#1\relax
\begin{enumerate}[label=\arabic*.] % 1.2.3.
\or
\begin{enumerate}[label=\textcircled{\arabic*}] % ①②③
\or
\begin{enumerate}[label=(\roman*)] % (i)(ii)(iii)
\else
\begin{enumerate}[label=\arabic*.] % 默认格式
\fi
}{
\end{enumerate}
}

\def\myIndex{0}
% \input{\path/cover_package_\myIndex.tex}

\def\myTitle{数学分析笔记}
\def\myAuthor{Zhang Liang}
\def\myDateCover{\today}
\def\myDateForeword{\today}
\def\myForeword{前言标题}
\def\myForewordText{
    前言内容
}
\def\mySubheading{副标题}


\begin{document}
	% \input{\path/cover_text_\myIndex.tex}

\newpage
\thispagestyle{empty}
\begin{center}
    \Huge\textbf{\myForeword}
\end{center}
\myForewordText
\begin{flushright}
    \begin{tabular}{c}
        \myDateForeword
    \end{tabular}
\end{flushright}

\newpage
\pagestyle{plain}
\setcounter{page}{1}
\pagenumbering{Roman}
\tableofcontents

\newpage
\pagenumbering{arabic}
\setcounter{chapter}{0}
\setcounter{page}{0}

\pagestyle{fancy}
\fancyfoot[C]{\thepage}
\renewcommand{\headrulewidth}{0.4pt}
\renewcommand{\footrulewidth}{0pt}








	\else
	\fi
	%标题
	\chapter{附录}
	这一部分中,对于正文中因为逻辑结构无法提及的部分,进行补充。包括特殊函数,有趣的数学概念,一些命题的全新解法,以及难以推导的公式
	证明可能使用复分析、实分析、泛函等超纲内容
	%--------------------正文---------------------------
	\section{欧拉-麦克劳林公式}
		我们直观上会认为,一个连续的级数和应该可以用这一整数区间的积分来大致模拟。一些结果也的确呈现了这个特征,比如以下有关调和级数的命题:
		
		$\sum\limits_{n=1}^{x} \frac{1}{n} = \int_{1}^{x} \frac{1}{t}\d t +o(1)$
		
		特别地,其中的余项有$o(1)\to\varepsilon$,$\varepsilon$为欧拉常数.
		
		我们对这里的余项做探讨,探究何时可以积分和求和的差值为常数阶,以及这一余项大小如何。
		\subsection{伯努利数}
		\begin{para}{0}
			\point{定义}
			\begin{defn}{伯努利数的级数定义}{}
				符合以下级数展开的数列称为伯努利数:
				\begin{equation}
					\frac{z}{e^z -1} = \sum_{n=0}^{\infty}\frac{B_n}{n!}z^n
				\end{equation}
			\end{defn}
			以上的定义十分简洁,但是难以计算。但是显然,借助Taylor级数我们可以推导出等价的递归定义:
			\begin{defn}{伯努利数的递归定义}{}
				\begin{equation}
					B_0=1,B_n=-\frac{1}{n+1}\sum\limits_{k=0}^{n-1}\binom{n+1}{k}B_{k}
				\end{equation}
			\end{defn}
			\begin{proof}
				利用级数定义中的展开式反推:
				
				$\because\left(e^z-1\right)\left(\frac{z}{e^z-1}\right)=z$
				
				$\therefore z\left[\frac{1}{1!}+\frac{1}{2!}z+...\cdots+\frac{1}{(n+1)!}z^n+R_1(z)\right]\left[B_0+B_1 z+\cdots+\frac{B_n}{n!}z^n+R_2(z)\right]=z$
				
				对比常数项可知:$B_0=1$
				进一步,如果n大于1,对比系数可得:
				$B_0\cdot\frac{1}{(n+1)!}+B_1\cdot\frac{1}{n!}+\cdots+B_k\cdot\frac{1}{k!}\cdot\frac{1}{(n-k)!}\cdots\frac{B_n}{n!}\cdot\frac{1}{1!}=0$
				
				左右同乘$(n+1)!$,得:
				$B_0\cdot\frac{(n+1)!}{(n+1)!\cdot 0!}+B_1\cdot\frac{(n+1)!}{n!\cdot 1!}+\cdots+B_k\frac{(n+1)!}{k!\cdot (n-k)!}+\cdots+B_n\cdot\frac{(n+1)!}{n!\cdot1!}=0$
				
				$\Rightarrow \sum\limits_{k=0}{n} \binom{n+1}{k}B_k=0$
				
				$\Rightarrow \sum\limits_{k=0}{n-1} \binom{n+1}{k}B_k=-(n+1)B_n$
				
				$\Rightarrow B_n=-\frac{1}{n+1}\sum\limits_{k=0}^{n-1}\binom{n+1}{k}B_{k}$
				
			\end{proof}
			\point{性质}
				\begin{para}{1}
					\point{除1外,所有奇数项的伯努利数为0}
					\begin{proposition}
						$B_{2n+1}=\begin{cases}
								0,n\geqslant 1 \\
								-\frac{1}{2},n=0 
							\end{cases}$
					\end{proposition}
					\begin{proof}
						由前面的伯努利数定义可知,$f(z)=\frac{z}{e^z-1}$在$\C-\{0\}$上解析
						
						考虑函数$\phi(x)=\frac{f(z)}{z^{2n+2}}$的洛朗展开
						
						$\phi(z)=\frac{B_0}{0!}z^{-2n-2}+\cdots+\frac{B_{2n+1}}{(2n+1)!}z^{-1}+\cdots$
						
						$\Rightarrow Res_{z=0}\phi(z)=\frac{B_{2n+1}}{(2n+1)!}$
						
						$\Rightarrow \oint_{C} \phi(z) \d z = 2\pi i\cdot \frac{B_{2n+1}}{(2n+1)!}$,其中$C=\{|z|=1\}$
						
						记上述积分为$I$,做代换$z=e^{i\theta},\theta\in[0,2\pi]$
						
						$I=\int_{0}^{2\pi} i e^{i\theta}\frac{e^{i\theta}}{e^{e^{i\theta}}-1}e^{-(2n+2)i\theta} \d \theta$
						
						$=\int_{0}^{2\pi} \frac{e^{-2ni\theta}}{e^{e^{i\theta}}-1}\d\theta$
						
						$=\int_{0}^{\pi} \frac{e^{-2ni\theta}}{e^{e^{i\theta}}-1}\d\theta+\int_{\pi}^{2\pi} \frac{e^{-2ni\theta}}{e^{e^{i\theta}}-1}\d\theta$
						
						$=\int_{0}^{\pi} \frac{e^{-2ni\theta}}{e^{e^{i\theta}}-1}\d\theta+\int_{\pi}^{2\pi} \frac{e^{-2ni\theta+2n\pi i}}{e^{e^{i\theta}}-1}\d\theta$
						
						作代换$\alpha = \theta-\pi$,并利用标准指数函数在虚轴的半个周期上变号,得到:
						
						$=\int_{0}^{\pi} \frac{e^{-2ni\theta}}{e^{e^{i\theta}}-1}\d\theta+\int_{0}^{\pi} \frac{e^{-2ni\alpha+2n\pi i}}{e^{-e^{i\alpha}}-1}\d\alpha$
						
						$=\int_{0}^{\pi} -e^{-2ni\theta}=\begin{cases}
							0,n \geqslant 0 \\
							-\pi,n=0
						\end{cases}$
						
						$\Rightarrow B_{2n+1}=\begin{cases}
							0,n\geqslant 1 \\
							-\frac{1}{2},n=0 
						\end{cases}$
					\end{proof}
					\point{自然幂指数和的通式}
					\begin{proposition}
						$S_p(n)=\sum\limits_{k=1}^{n}k^p=\frac{1}{p+1}\sum\limits_{i=0}^{p} (-1)^i B_i \binom{p+1}{i} n^{p+1-i}$
					\end{proposition}
					\begin{proof}
						明天再说,累了
					\end{proof}
					\point{}
					\begin{proposition}
						$B_n=-n\zeta(1-n),n\geqslant 1$
					\end{proposition}
					
					\begin{proof}
						黎曼Zeta函数的定义为;
						
						$\zeta(s)=\begin{cases}
							\sum\limits_{k=1}^{\infty}\frac{1}{k^s},Re(s)>1 \\
							\frac{1}{1-2^{1-s}}\sum\limits_{k=1}^{\infty}\frac{(-1)^{k+1}}{k^s},0< Re(s) \leqslant 1 \\
							2^s \pi^{s-1} \sin\left(\frac{\pi s}{2}\right)\Gamma(1-s)\zeta(1-s),Re(s)\leqslant 0
						\end{cases}$
						
						解析延拓过程可以参考《复分析笔记》
						
						先考虑$n=2k+1,k>0$的情形,此时依Zeta函数定义,$\zeta(-2k)=0$,因为$\sin(n\pi)=0$,命题成立。
						
						而当$n=1$,$\zeta(0)=-\frac{1}{2},B_1=-\frac{1}{2}$,命题成立。
						
						接下来考虑n为偶数的情形:
						
						我们注意到,在余切函数的展开式中,恰好包含求和、位于分母的n这两种形式,通过Taylor展开可以组成我们期望的Zeta函数形式:
						
						
					\end{proof}
				\end{para}
			\begin{them}{欧拉-麦克劳林公式}{}
				假设$f(x)$无穷阶可导,那么:
				\begin{equation}
					\sum\limits_{a\leq n<b}f(n) = \int_{a}^{b}f(x)\d x+\sum\limits_{k=1}^{\infty}\frac{B_k}{k!}f^{(k-1)}(x)\big|_{a}^{b}+{(-1)}^{m}\int_{a}^{b}\frac{B_m({x})}{m!}f^{(m)}(x)\d x
				\end{equation}
				
			\end{them}
		\end{para}
		\subsection{伯努利多项式}
		\begin{para}{0}
			\point{定义}
			\begin{defn}{伯努利多项式的级数定义}{}
				满足以下级数的函数族$B_n(x)$称为伯努利多项式
				\begin{equation}
					\frac{z e^{xz}}{e^z-1}=\sum\limits_{n=0}^{\infty} \frac{B_n(x)}{n!}z^n
				\end{equation}
			\end{defn}
		\end{para}
	
	%附录:其他四种三角函数的泰勒展开
	\section{其他四种三角函数的泰勒展开}
		\subsection{$\tan x$}
			\begin{them}{正切函数的Taylor展开}{}
				\begin{equation}
					\tan x = \sum\limits_{n=1}^{\infty} \frac{(-1)^{n-1}2^{2n}(2^{2n}-1)B_{2n}}{(2n)!}x^{2n-1}
				\end{equation}
			\end{them}
			\begin{proof}
				怎么开始套娃了(恼)
			\end{proof}
	
	%附录:不定积分初等性判定
	\section{原函数初等性的判定方法}
		\subsection{切比雪夫定理}
			\begin{them}{切比雪夫定理}{}
				设$m,n,p\in \Q-\{0\}$,那么以下积分
				\begin{equation}
					\int x^m(a+bx^n)^p \d x
				\end{equation}
				初等的充要条件是:$p,\frac{m+1}{n},\frac{m+1}{n}+p$中至少有一个为整数
			\end{them}
		\subsection{刘维尔定理}
	
	%附录:超越积分的特殊解法
	\section{一些超越积分的特殊解法}
		\subsection{Direchlet积分}
		
	
	\ifx\allfiles\undefined
\end{document}
\fi