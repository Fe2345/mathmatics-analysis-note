\ifx\allfiles\undefined
\documentclass[12pt, a4paper, oneside, UTF8]{ctexbook}
\setCJKmainfont{SimSun}
\def\path{../config}
\usepackage{amsmath}
\usepackage{amsthm}
\usepackage{amssymb}
\usepackage{graphicx}
\usepackage{mathrsfs}
\usepackage{enumitem}
\usepackage{geometry}
\usepackage[colorlinks, linkcolor=black]{hyperref}
\usepackage{stackengine}
\usepackage{yhmath}
\usepackage{extarrows}
\usepackage{unicode-math}
\usepackage{tikz}
\usepackage{tikz-cd}

\usepackage{fancyhdr}
\usepackage[dvipsnames, svgnames]{xcolor}
\usepackage{listings}

\definecolor{mygreen}{rgb}{0,0.6,0}
\definecolor{mygray}{rgb}{0.5,0.5,0.5}
\definecolor{mymauve}{rgb}{0.58,0,0.82}

\graphicspath{ {figure/},{../figure/}, {config/}, {../config/} }

\linespread{1.6}

\geometry{
    top=25.4mm, 
    bottom=25.4mm, 
    left=20mm, 
    right=20mm, 
    headheight=2.17cm, 
    headsep=4mm, 
    footskip=12mm
}

\setenumerate[1]{itemsep=5pt,partopsep=0pt,parsep=\parskip,topsep=5pt}
\setitemize[1]{itemsep=5pt,partopsep=0pt,parsep=\parskip,topsep=5pt}
\setdescription{itemsep=5pt,partopsep=0pt,parsep=\parskip,topsep=5pt}

\lstset{
    language=Mathematica,
    basicstyle=\tt,
    breaklines=true,
    keywordstyle=\bfseries\color{NavyBlue}, 
    emphstyle=\bfseries\color{Rhodamine},
    commentstyle=\itshape\color{black!50!white}, 
    stringstyle=\bfseries\color{PineGreen!90!black},
    columns=flexible,
    numbers=left,
    numberstyle=\footnotesize,
    frame=tb,
    breakatwhitespace=false,
} 
\usepackage[strict]{changepage} 
\usepackage{framed}
\usepackage{tcolorbox}
\tcbuselibrary{most}

\definecolor{greenshade}{rgb}{0.90,1,0.92}
\definecolor{redshade}{rgb}{1.00,0.88,0.88}
\definecolor{brownshade}{rgb}{0.99,0.95,0.9}
\definecolor{lilacshade}{rgb}{0.95,0.93,0.98}
\definecolor{orangeshade}{rgb}{1.00,0.88,0.82}
\definecolor{lightblueshade}{rgb}{0.8,0.92,1}
\definecolor{purple}{rgb}{0.81,0.85,1}

% #### 将 config.tex 中的定理环境的对应部分替换为如下内容
% 定义单独编号,其他四个共用一个编号计数 这里只列举了五种,其他可类似定义(未定义的使用原来的也可)
\newtcbtheorem[number within=section]{defn}%
{定义}{colback=OliveGreen!10,colframe=Green!70,fonttitle=\bfseries}{def}

\newtcbtheorem[number within=section]{lemma}%
{引理}{colback=Salmon!20,colframe=Salmon!90!Black,fonttitle=\bfseries}{lem}

% 使用另一个计数器 use counter from=lemma
\newtcbtheorem[use counter from=lemma, number within=section]{them}%
{定理}{colback=SeaGreen!10!CornflowerBlue!10,colframe=RoyalPurple!55!Aquamarine!100!,fonttitle=\bfseries}{them}

\newtcbtheorem[use counter from=lemma, number within=section]{criterion}%
{准则}{colback=green!5,colframe=green!35!black,fonttitle=\bfseries}{cri}

\newtcbtheorem[use counter from=lemma, number within=section]{corollary}%
{推论}{colback=Emerald!10,colframe=cyan!40!black,fonttitle=\bfseries}{cor}
% colback=red!5,colframe=red!75!black

% 这个颜色我不喜欢
%\newtcbtheorem[number within=section]{proposition}%
%{命题}{colback=red!5,colframe=red!75!black,fonttitle=\bfseries}{cor}

% .... 命题 例 注 证明 解 使用之前的就可以(全文都是这种框框就很丑了),也可以按照上述定义 ...
\renewenvironment{proof}{\par\textbf{证明:}\;}{\qed\par}
\newenvironment{solution}{\par{\textbf{解:}}\;}{\qed\par}
\newtheorem{proposition}{\indent 命题}[section]
\newtheorem{example}{\indent \color{SeaGreen}{例}}[section] % 绿色文字的 例 ,不需要就去除\color{SeaGreen}{}
\newtheorem*{rmk}{\indent 注}
\usepackage{amssymb}
\def\d{\mathrm{d}}
\def\R{\mathbb{R}}
\def\C{\mathbb{C}}
\def\Q{\mathbb{Q}}
\newcommand{\bs}[1]{\boldsymbol{#1}}
\newcommand{\ora}[1]{\overrightarrow{#1}}
\newcommand{\myspace}[1]{\par\vspace{#1\baselineskip}}
\newcommand{\xrowht}[2][0]{\addstackgap[.5\dimexpr#2\relax]{\vphantom{#1}}}
\newenvironment{ca}[1][1]{\linespread{#1} \selectfont \begin{cases}}{\end{cases}}
\newenvironment{vx}[1][1]{\linespread{#1} \selectfont \begin{vmatrix}}{\end{vmatrix}}
\newcommand{\tabincell}[2]{\begin{tabular}{@{}#1@{}}#2\end{tabular}}
\newcommand{\pll}{\kern 0.56em/\kern -0.8em /\kern 0.56em}
\newcommand{\dive}[1][F]{\mathrm{div}\;\bs{#1}}
\newcommand{\rotn}[1][A]{\mathrm{rot}\;\bs{#1}}
\usepackage{xeCJK}
\setCJKmainfont{SimSun}[BoldFont={SimHei}, ItalicFont={KaiTi}] % 设置中文支持

\newcommand{\point}[1]{\item {#1}}
\newenvironment{para}[1]{%
\ifcase#1\relax
\begin{enumerate}[label=\arabic*.] % 1.2.3.
\or
\begin{enumerate}[label=\textcircled{\arabic*}] % ①②③
\or
\begin{enumerate}[label=(\roman*)] % (i)(ii)(iii)
\else
\begin{enumerate}[label=\arabic*.] % 默认格式
\fi
}{
\end{enumerate}
}

\def\myIndex{0}
% \input{\path/cover_package_\myIndex.tex}

\def\myTitle{数学分析笔记}
\def\myAuthor{Zhang Liang}
\def\myDateCover{\today}
\def\myDateForeword{\today}
\def\myForeword{前言标题}
\def\myForewordText{
    前言内容
}
\def\mySubheading{副标题}


\begin{document}
	% \input{\path/cover_text_\myIndex.tex}

\newpage
\thispagestyle{empty}
\begin{center}
    \Huge\textbf{\myForeword}
\end{center}
\myForewordText
\begin{flushright}
    \begin{tabular}{c}
        \myDateForeword
    \end{tabular}
\end{flushright}

\newpage
\pagestyle{plain}
\setcounter{page}{1}
\pagenumbering{Roman}
\tableofcontents

\newpage
\pagenumbering{arabic}
\setcounter{chapter}{0}
\setcounter{page}{0}

\pagestyle{fancy}
\fancyfoot[C]{\thepage}
\renewcommand{\headrulewidth}{0.4pt}
\renewcommand{\footrulewidth}{0pt}








	\else
	\fi
	%标题
	\chapter{附录}
	这一部分中,对于正文中因为逻辑结构无法提及的部分,进行补充。包括特殊函数,有趣的数学概念,一些命题的全新解法,以及难以推导的公式
	证明可能使用复分析、实分析、泛函等超纲内容
	%--------------------正文---------------------------
	
	%附录:实数的等价定义
	\section{实数的等价定义}
	第二章提及的实数公理定义了实数;但是,其实还有一些构造性方法可以定义实数。戴德金分割和柯西序列是其中的两个。
	
	首先,我们要先构造出自然数集,进而构造有理数集。
		\subsection{Peano公理,自然数集}
			\begin{para}{0}
				\point{自然数集的定义}
					\begin{defn}{自然数集的Peano公理}{}
						存在一个运算$*:\N \rightarrow \N_+$,称为后继,满足以下性质:
						
						\ding{172} 存在一个元素$0$,$0 \in \N$
						
						\ding{173} $\forall a \in \N ,a^* \neq 0$
						
						\ding{174} 如果$a \neq b,$那么$a^* \neq b^*$
						
						这个集合$\N$称为自然数集,$\N_+$称为正整数集。
					\end{defn}
					可以看出来,Peano公理与实数公理不同,是借助后继运算而不是加法和乘法。不过,接下来我们将递归地定义加法和乘法,并定义自然数的序。
					
					我们按照之前的常用记号,记$0^*=1,1^*=2,2^*=3,\cdots$
				\point{加法}
					\begin{defn}{自然数的加法}{}
						我们定义运算$+:\N \times \N \rightarrow \N$,称为加法,满足以下性质:
						
						$a^*+b=(a+b)^*$
						
						特别地,我们定义$0+b=b$
					\end{defn}
					我们首先验证此处定义的加法是否依旧具有交换律和结合律
					
					我们首先证明几个引理。
					\begin{lemma}{}{}
						$n+0=n$
					\end{lemma}
					\begin{proof}
						对n使用数学归纳法:首先,当$n=0,0+0=0$依定义是成立的。
						
						接下来假设$n+0=n$,那么,$n^*+0=(n+0)^*=n^*$,依归纳原理,命题得证。
					\end{proof}
					\begin{lemma}{}{}
						$n+m^{*}=(n+m)^{*}$
					\end{lemma}
					\begin{proof}
						固定$m$,对$n$作数学归纳法:
						
						首先,当$n=0$,$0+m^*=m^*=(0+m)^*$
						
						现在假设$n+m^*=(n+m)^*$
						
						那么,$n^*+m^*=\left(n^*+m\right)^*$
						
						这里可以将$n^*$视为一个整体,加上$m$的后继,于是命题得证。
					\end{proof}
					于是,我们证明了自然数加法的交换律:
					\begin{proposition}
						$n+m=m+n$
					\end{proposition}
					\begin{proof}
						固定$n$,对$m$作数学归纳法。
						
						首先,当$m=0$,$n+0=0+n$,命题成立。
						
						现在假设$n+m=m+n$,我们只需证明$n+m^*=m^*+n$,而这是已经证明的了。
						
						但是,我们已经知道:$n+m^*=(n+m)^*,m^*+n=(m+n)^*$,而且我们已经假设$n+m=m+n$,于是命题得证。
					\end{proof}
					在证明过程中可以看出以下推论:
					\begin{corollary}{}{}
						$n+m^*=n^*+m$
					\end{corollary}
					于是有以下推论:
					\begin{corollary}{}{}
						$n^*=n+1$
					\end{corollary}
					\begin{proof}
						在上面的推论中代入$m=0$
						
						$n+0^*=n^*+0$
						
						$\Rightarrow n+1=n^*$
					\end{proof}
					此后,我们不再使用$n^*$的记号,而是用$n+1$代替。
					\begin{proposition}
						$(a+b)+c=a+(b+c)$
					\end{proposition}
					\begin{proof}
						固定$a,c$,对$b$作数学归纳法。
						
						当$b=0$,$(a+0)+c=a+c=a+(0+c)$,命题成立。
						
						现在假设$(a+b)+c=a+(b+c)$
						
						那么,$\left(a+(b+1)\right)+c$
						
						$=\left((a+b)+1\right)+c=(a+b+c)+1$
						
						$=a+\left((b+c)+1\right)=a+\left((b+1)+c\right)$,命题得证。
					\end{proof}
					从此以后,我们就可以使用形如$a+b+c$的符号了,因为我们已经证明了这种求和在不同顺序下有相同的结果。
					
					至此,我们证明了加法在自然数集上的交换律和结合律。
					
					我们在证明一个显然的命题。
					\begin{proposition}{}
						$a = b \Leftrightarrow a+c=b+c$
					\end{proposition}
					\begin{proof}
						对$c$作数学归纳法:
						
						当$c=0$,$a+0=a,b+0=b$,所以命题显然成立。
						
						现在假设$a = b \Leftrightarrow a+c=b+c$
						
						那么$a=b\Leftrightarrow a+c=b+c \Leftrightarrow (a+c)+1=(b+c)+1 \Leftrightarrow a+(c+1)=b+(c+1)$,于是命题得证。
					\end{proof}
				\point{乘法}
					\begin{defn}{自然数的乘法}
						我们定义运算$\cdot : \N \times \N \rightarrow \N$,称为乘法,满足以下性质:
						
						$a \cdot (b+1) = a\cdot b + a$
						
						并且特别定义:$a \cdot 0 = 0$
					\end{defn}
					我们仿照加法,先证明几个引理。
					\begin{lemma}{}{}
						$0 \cdot a = 0$
					\end{lemma}
					\begin{proof}
						对$a$作数学归纳法:
						
						当$a=0$,$0 \cdot 0 = 0$
						
						现在假设$0 \cdot a = 0$,那么:$0 \cdot (a+1) = 0\cdot a + 0=0$,命题得证
					\end{proof}
					\begin{lemma}{}{}
						$(b+1)\cdot a = b\cdot a+a$
					\end{lemma}
					\begin{proof}
						固定$b$,对$a$作数学归纳法:
						
						当$a=0$,$(b+1)\cdot 0 = 0 = b\cdot 0+0$
						
						现在假设$(b+1)\cdot a = b\cdot a+a$
						
						那么,$(b+1)\cdot (a+1)=(b+1)\cdot a+(b+1)$
						
						$=b\cdot a+a+(b+1)=b\cdot a+b+(a+1)$
						
						$=b\cdot a + a + (b+1)=b\cdot (a+1)+(a+1)$,命题得证。
					\end{proof}
					接下来证明交换律
					\begin{proposition}{}{}
						$a\cdot b = b\cdot a$
					\end{proposition}
					\begin{proof}
						固定$a$,对$b$作数学归纳法。
						
						当$b=0$,$a \cdot 0 = 0 \cdot a$,命题成立。
						
						现在假设$a\cdot b = b\cdot a$,我们欲证$a\cdot (b+1)=(b+1)\cdot a$
						
						但是我们知道:$a\cdot (b+1)=a\cdot b+a,(b+1)\cdot a = b\cdot a+a$,并且我们已经假设$a\cdot b=b\cdot a$,于是命题成立。
					\end{proof}
					于是有以下推论:
					\begin{corollary}{}{}
						$a\cdot 1 = 1\cdot a = a$
					\end{corollary}
					\begin{proof}
						$a\cdot 1 = a\cdot (0+1)=a\cdot 0+a=a$,再结合交换律即得证。
					\end{proof}
					这个推论说明:我们利用Peano公理定义的1依旧是对乘法的单位元,与我们在实数公理下定义的1一致。
					
					接下来证明分配律。
					\begin{proposition}{}{}
						$a\cdot (b+c)=a\cdot b+a\cdot c$
					\end{proposition}
					\begin{proof}
						固定$a$和$b$,对$c$作数学归纳法。
						
						当$c=0$,$a\cdot (b+0)=a\cdot b=a\cdot b+a\cdot 0$,命题成立。
						
						现在假设$a\cdot (b+c)=a\cdot b+a\cdot c$
						
						那么,$a\cdot \left(b+(c+1)\right)=a\cdot \left((b+c)+1\right)$
						
						$=a\cdot (b+c)+a=a\cdot b+a\cdot c+a=a\cdot b+a\cdot (c+1)$,于是命题得证。
					\end{proof}
					最后,我们证明结合律。
					\begin{proposition}{}{}
						$a\cdot (b\cdot c)=(a\cdot b)\cdot c$
					\end{proposition}
					\begin{proof}
						固定$a,b$,对$c$作数学归纳法:
						
						当$c=0$,$a\cdot (b\cdot 0)=a\cdot 0=0=(a\cdot b)\cdot 0$,命题成立
						
						现在假设$a\cdot (b\cdot c)=(a\cdot b)\cdot c$
						
						那么,$a\cdot \left(b\cdot (c+1)\right)=a\cdot (b\cdot c+b)$
						
						$=a\cdot (b\cdot c)+a\cdot b=(a\cdot b)\cdot c+a\cdot b=(a\cdot b)\cdot (c+1)$,命题得证。
					\end{proof}
					至此,我们证明了自然数集上关于加法、乘法、加法与乘法联系的公理。
				\point{序}
					\begin{defn}{自然数集上的序}{}
						我们定义:$\N$上存在一个关系$\geqslant$,称为大于等于,满足以下性质:
						
						$a \geqslant b$,当且仅当$\exists p \in \N,a=b+p$
						
						如果$a \geqslant b$,我们又称$b \leqslant a$,如此定义的关系$\leqslant$称为小于等于;
						
						如果$a \geqslant b$,并且$a \neq b$,那么我们称$a > b$,如此定义的关系$>$称为大于;
						
						如果$a \leqslant b$,并且$a \neq b$,那么我们称$a < b$,如此定义的关系$<$称为小于
					\end{defn}
					我们首先验证这里的序的确是序公理中的序,即验证自反性、传递性、反对称性。
					\begin{proposition}
						$\forall a\in \N,a \geqslant a$
					\end{proposition}
					\begin{proof}
						注意到,$a=a+0,0\in\N$
					\end{proof}
					\begin{proposition}
						$a \geqslant b,b\geqslant c \Rightarrow a\geqslant c$
					\end{proposition}
					\begin{proof}
						$a \geqslant b,b\geqslant c \Leftrightarrow \exists s,t\in \N,a=b+s,b=c+t$
						
						即$a=c+s+t$,而$s+t\in\N$
					\end{proof}
					\begin{proposition}
						$a \geqslant b,b\geqslant a \Rightarrow a=b$
					\end{proposition}
					\begin{proof}
						$a \geqslant b,b\geqslant a \Rightarrow \exists s,t \in \N,a=b+s,b=a+t$
						
						$\Rightarrow s+t=0$,现在假设$s\neq 0$,那么$\exists u\in\N,u+1=s$
						
						所以$(u+1)+t=0\Rightarrow (u+t)+1=0$,但是按照公理,应该没有元素的后继为$0$。
						
						所以我们有$s=0$,此时$a=b+0=b$。
					\end{proof}
					接下来我们证明序与加法联系的公理在现在依旧是成立的:
					\begin{proposition}
						$a \geqslant b \Rightarrow \forall s\in\N,a+s \geqslant b+s$
					\end{proposition}
					\begin{proof}
						$a \geqslant b \Rightarrow \exists t\in\N,a=b+t$
						
						只需证明,$\forall s \in \N a+s=b+t+s$成立,但是按照之前我们证明的命题,这是显然的。
					\end{proof}
			\end{para}
		\subsection{整数集}
			\begin{para}{0}
				\point{整数集的定义}
					\begin{defn}{整数集}{}
						我们将形如$a\textemdash b$的数对称为整数,其中$a,b \in \N$。
						
						并特别规定:$a\textemdash b = c\textemdash d \Leftrightarrow a+d=b+c$
						
						集合$\Z = \{a\textemdash b| a,b \in \N\}$称为整数集
					\end{defn}
					我们还希望,整数集是自然数集的一个补充,因此有以下补充定义:
					\begin{defn}{自然数集和整数集}{}
						我们定义:$a\textemdash 0 = a,a\in\N$
					\end{defn}
					这个定义是良好的,因为我们容易知道,$\{a\textemdash 0 | a\in\N\}$与$\N$同构。
					
					整数集构造的想法是很自然的:整数集在直观的认知下是自然数集及其“相反数”的并,因此我们希望我们定义的整数可以分解为两个自然数的“差”。
					
					接下来我们沿着这个想法,定义出我们直观上的加、乘、负、减这三种运算。
				\point{整数的加法}
					\begin{defn}{整数的加法}{}
						我们定义:
						
						$(a\textemdash b)+(c\textemdash d) := (a+c)\textemdash(b+d)$
					\end{defn}
					我们先验证加法的定义是良好的,即如果按照整数的定义替换整数的形式,加法的结果不会改变。
					\begin{proposition}{}
						如果$(a_1\textemdash b_1)=(a_2\textemdash b_2)$,那么一定有
						
						$(a_1\textemdash b_1)+(c\textemdash d)=(a_2\textemdash b_2)+(c\textemdash d),(c\textemdash d)+(a_1\textemdash b_1)=(c\textemdash d)+(a_2\textemdash b_2)$
					\end{proposition}
					\begin{proof}
						先证第一条。$(a_1\textemdash b_1)+(c\textemdash d)=(a_1+c)\textemdash(b_1+d),(a_2\textemdash b_2)+(c\textemdash d)=(a_2+c)\textemdash(b_2+d)$,
						
						于是只需证$a_1+c+b_2+d=b_1+d+a_2+c$,但这按照整数相等的定义是显然的。
						
						同理可证第二条。
					\end{proof}
					在定义了整数的加法后,我们还需验证,在加法的关系下$a \textemdash 0$依旧和$a$是等同的。
					\begin{proposition}
						$(a\textemdash 0)+(b\textemdash 0) = a+b = (a+b)\textemdash 0$
					\end{proposition}
					这个命题是显然的。
					
					接下来我们验证之前自然数的运算现在依旧是成立的。
					\begin{proposition}
						$a+b=b+a$
					\end{proposition}
					\begin{proof}
						设$a = (a_1 \textemdash a_2),b=(b_1 \textemdash b_2)$,
						
						于是,$a+b = (a_1+b_1 \textemdash a_2+b_2)$
						
						$b+a = (b_1+a_1 \textemdash b_2+a_2)$,但是我们已经在自然数集中证明了交换律。
					\end{proof}
					\begin{proposition}
						$(a+b)+c=a+(b+c)$
					\end{proposition}
					\begin{proof}
						设$a = (a_1 \textemdash a_2),b=(b_1 \textemdash b_2),c=(c_1 \textemdash c_2)$,
						
						于是,$(a+b)+c = \left((a_1+b_1)+c_1 \textemdash (a_2+b_2)+c_2\right)$
						
						$a+(b+c) = \left(a_1+(b_1+c_1) \textemdash a_2+(b_2+c_2)\right)$,但是我们已经在自然数集中证明了结合律。
					\end{proof}
					\begin{proposition}
						$x+0=0+x=x$
					\end{proposition}
					\begin{proof}
						因为已经证明交换律,所以只需证明$x+0=x$,但是这是显然的。
					\end{proof}
				\point{整数的乘法}
					\begin{defn}{整数的乘法}{}
						我们定义:
						
						$(a\textemdash b)\cdot(c\textemdash d) := (a\cdot c + b\cdot d)\textemdash(a\cdot d+b\cdot c)$
					\end{defn}
					同样地,我们先验证乘法的定义是良好的。
					\begin{proposition}
						如果$(a_1 \textemdash b_1)=(a_2 \textemdash b_2)$,
						
						那么$(a_1 \textemdash b_1)\cdot (c\textemdash d)=(a_2 \textemdash b_2)\cdot (c\textemdash d),(c\textemdash d)\cdot (a_1 \textemdash b_1)=(c\textemdash d)\cdot(a_2 \textemdash b_2)$
					\end{proposition}
					\begin{proof}
						先证第一条。
						
						$(a_1 \textemdash b_1)\cdot (c\textemdash d)=(a_1 \cdot c+b_1\cdot d)\textemdash (a_1\cdot d+b_1\cdot c)$
						
						$(a_2 \textemdash b_2)\cdot (c\textemdash d)=(a_2\cdot c+b_2\cdot d)\textemdash (a_2\cdot d+b_2\cdot c)$
						
						那么只需证明:$a_1 \cdot c+b_1\cdot d+a_2\cdot d+b_2\cdot c=a_1\cdot d+b_1\cdot c+a_2\cdot c+b_2\cdot d$
						
						$\Leftrightarrow (a_1+b_2)\cdot c + (a_2+b_1)\cdot d = (b_1+a_2)\cdot c+(a_1+b_2)\cdot d$,但是按照整数相等的定义这是显然的。
						
						同理可证第二条。
					\end{proof}
				\point{整数的负}
					\begin{defn}{整数的负}{}
						我们定义整数$(a \textemdash b)$的负为:
						
						$-(a\textemdash b) := (b\textemdash a)$,记作$-(a\textemdash b)$
					\end{defn}
			\end{para}
		\subsection{有理数集}
		\subsection{戴德金分割}
		\subsection{柯西序列}
	%附录:R,R^m,R^infty的比较
	\section{\texorpdfstring{$\R,\R^m,\R^\infty$}势的比较,皮亚诺曲线}
		\subsection{约定的记号}
			本节中引入以下全新的记号:
			\begin{para}{0}
				\point{$X^Y$}
					\begin{defn}{}{}
						$X^Y := \{f| f:Y \rightarrow X\}$
						
						即$Y$到$X$的全体映射的集合
					\end{defn}
				\point{$X^\infty$}
					\begin{defn}{}{}
						$X^\infty := \{\{a_n\}| a_i\in X\}$
						
						即元素为$X$中元素的全体点列的集合
					\end{defn}
					可见,$X^\infty$就是$X^Y$在$Y=\N$下的特例,也就是$X^{\infty}=X^{\N}$
				\point{$\mathscr{P}(A)$}
					\begin{defn}{}{}
						$\mathscr{P}(A) = \{V|V \subseteq A\}$
						
						即$A$的全体子集组成的集合
					\end{defn}
			\end{para}
			
		\subsection{\texorpdfstring{$\R,\R^m,\R^\infty$}势的比较}
			我们将证明:$Card \R = Card \R^m = Card \R^\infty$
			
			我们的证明过程会按照此交换图顺序。
			\[\begin{tikzcd}
				&& {(0,1)} \\
				& {\mathbb{R}} & {[0,1]} & {\{0,1\}^{\infty}} & {\mathscr{P}(\mathbb{N})} \\
				{\mathbb{R}^m} & {\mathbb{R}^\infty} & {[0,1]^\infty} & {\left(\{0,1\}^{\infty}\right)^\infty} & {\mathscr{P}(\mathbb{Q})}
				\arrow["{Lemma2}", from=1-3, to=2-3]
				\arrow["{Lemma1}", from=2-2, to=1-3]
				\arrow["{Proposition2}"', from=2-2, to=3-1]
				\arrow["{Proposition1}", leftrightarrow, from=2-2, to=3-2]
				\arrow["{Lemma3}"', from=2-3, to=2-4]
				\arrow["{Lemma4}", from=2-4, to=2-5]
				\arrow["{Lemma5}", from=2-5, to=3-5]
				\arrow["{Proposition2}", from=3-2, to=3-1]
				\arrow["{Lemma6}", from=3-3, to=3-2]
				\arrow["{Lemma6}", from=3-4, to=3-3]
				\arrow["{Lemma4}", from=3-5, to=3-4]
			\end{tikzcd}\]
			
			\begin{lemma}{引理1}{}
				$Card \R = Card (0,1)$
			\end{lemma}
			\begin{proof}
				考虑以下映射$f(x)=\frac{1}{1+e^{-x}}$
				
				这个函数显然单调增,所以是单射;满射通过单调性和$\lim\limits_{x\to - \infty}=0,\lim\limits_{x\to +\infty}=1$可知
			\end{proof}
			\begin{lemma}{引理2}{}
				$Card (0,1) = Card [0,1]$
			\end{lemma}
			\begin{proof}
				考虑以下$[0,1]\rightarrow(0,1)$映射$f(x)=\begin{cases}
					\frac{1}{2},x=0 \\
					\frac{1}{3},x=1 \\
					\frac{1}{n+2},x=\frac{1}{n},n>1 \\
					x,x\neq 0,x\neq 1,x\neq \frac{1}{n},n>1
					\end{cases}$
					
				显然,这个映射是一个$[0,1]$到$(0,1)$的双射
			\end{proof}
			这个引理证明中映射的构造采取了“希尔伯特的旅馆”的想法。这个想法是说:
			
			如果有一个具有可数个房间并且已经满了的旅馆,我们希望在不赶走任意一个顾客的前提下,将一个新的顾客安排进旅馆。我们可以让每个顾客都来到自己的下一个房间,而将新的顾客安排在第一个房间。
			
			用数学语言描述就是:如果有一个点列$\{a_n\}$,我们希望在不舍弃任何一项下添加$m$项,我们只需要将原来的$a_i$迁移到$a_{i+m}$,而把新的项添加到现在空缺的前$m$项。
			
			在证明过程中,我们采取的是点列$\{\frac{1}{n},n>1\}$,将$0,1$迁移到这个序列内。
			\begin{lemma}{引理3}{}
				$Card [0,1] = Card \{0,1\}^\infty$
			\end{lemma}
			\begin{proof}
				从第一章中关于进制的定理可知,$\forall a \in [0,1],\exists \{a_n\},a_i \in \{0,1\}$,使得
				
				$a = \sum\limits_{i=0}^{\infty}\frac{a_i}{2^i}$,且这一对应关系是双射的。
			\end{proof}
			\begin{lemma}{引理4}{}
				$Card \{0,1\}^S = Card \mathscr{P}(S)$
			\end{lemma}
			\begin{proof}
				定义示性函数$\chi_A:S\rightarrow\{0,1\}$。
				
				$\chi_A(x) = \begin{cases}
					1,x \in A \\
					0,x \in S-A
				\end{cases}$
				
				并定义映射$\varphi : \mathscr{P}(S)\rightarrow \{0,1\}^S$
				
				$\varphi(A) := \chi_A$
				
				接下来证明$\varphi$是双射:
				
				$\varphi$的单射性是显然的;而对于满射性,$\forall f \in \{0,1\}^S$
				
				令$A = \{x|f(x)=1\}$,那么就有$\chi_A=f$,双射性得证
			\end{proof}
			\begin{lemma}{引理5}
				集合$A,B$若有$Card A = Card B$,那么有$Card \mathscr{P}(A) = Card \mathscr{P}(B)$
			\end{lemma}
			\begin{proof}
				设$Card A=Card B$,且$A\rightarrow B$的双射是$f$
				
				那么我们构造:$g:\mathscr{P}(A)\rightarrow\mathscr{P}(B)$
				
				$g(X) := \{b|b=f(a),a\in X\}$
				
				接下来证明$g$是双射
				
				先证明单射性。设$X\neq Y$,所以一定$\exists a$,只处于$X,Y$这两个集合中的一个
				
				因为$f$是一个双射,所以$f(a)$只处于$g(X),g(Y)$这两个集合中的一个,所以$g(X)\neq g(Y)$,单射性得证。
				
				对于满射性,由$f$的满射性可知这是显然的。
			\end{proof}
			\begin{lemma}{引理6}
				如果$Card X = Card Y$,那么$Card X^Z = Card Y^Z$
			\end{lemma}
			\begin{proof}
				设$X\rightarrow Y$的双射为$f$
				
				那么定义映射$\varphi:X^Z \rightarrow Y^Z$
				
				$\varphi(g) = f \circ g$
				
				$\varphi$的双射性是显然的。
			\end{proof}
			\begin{proposition}
				$Card \R = Card \R^\infty$
			\end{proposition}
			\begin{proof}
				由引理1,2,3可知,$Card \R = Card \{0,1\}^\infty$
				
				接下来,只需注意到两个事实:$Card {\left(\{0,1\}^{\infty}\right)^{\infty} }= Card \{0,1\}^{\Q}$,因为我们知道$Card {\N^2} = Card {\Q}$,
				
				并且注意到$Card \N = Card \Q$,再利用引理4,5,即得$Card \R = Card \left(\{0,1\}^\infty\right)^\infty$
				
				接下来,我们利用引理3的结果:$Card [0,1] = Card \{0,1\}^\infty$,并且使用引理6,我们即得:
				
				$Card \R = Card [0,1]^\infty$
				
				随后,再次利用引理6,以及引理1,2得到的结果:$Card \R = Card [0,1]$
				
				我们至此就得到我们所需结果:$Card \R = Card \R^\infty$
			\end{proof}
			\begin{proposition}
				$Card \R = Card \R^m$
			\end{proposition}
			\begin{proof}
				设$R_1 = \{(a,0,\cdots)|a \in \R\}$
				
				$R_2 = \{(a_1,a_2,\cdots,a_m,\cdots)|a_i \in \R\}$
				
				显然,$R_1 \subset R_2 \subset \R^\infty$
				
				但是$Card R_1 = Card \R,Card R_2 = Card \R^m,Card \R = Card \R^\infty$,那么一定有$Card \R = Card \R^m$
			\end{proof}
		\subsection{皮亚诺曲线}
	%附录:伯努利数和欧拉数
	\section{伯努利数与欧拉数}
		在对泰勒公式的研究中,我们推到了标准指数函数$\exp x$的级数展开,它的系数是极其容易推导的。
		
		但是,如果我们考察其倒数,那么就变得相当难以推导:因为它的高阶导数一阶比一阶难以求得。
		
		这一附录中我们反其道而行之,先假设系数,再利用其与指数函数乘积求出系数,从而提出了两个重要的数列:伯努利数和欧拉数。
		
		特别地,由于三角函数的分析定义正是利用的指数函数,利用伯努利数和欧拉数可以求出很多之前难以求出的展开式。
		
		\subsection{伯努利数}
			\begin{para}{0}
				\point{定义}
				\begin{defn}{伯努利数的级数定义}{}
					符合以下级数展开的数列称为伯努利数:
					\begin{equation}
						\frac{z}{e^z -1} = \sum_{n=0}^{\infty}\frac{B_n}{n!}z^n
					\end{equation}
				\end{defn}
				以上的定义十分简洁,但是难以计算。但是显然,借助Taylor级数我们可以推导出等价的递归定义:
				\begin{defn}{伯努利数的递归定义}{}
					\begin{equation}
						B_0=1,B_n=-\frac{1}{n+1}\sum\limits_{k=0}^{n-1}\binom{n+1}{k}B_{k}
					\end{equation}
				\end{defn}
				\begin{proof}
					利用级数定义中的展开式反推:
					
					$\because\left(e^z-1\right)\left(\frac{z}{e^z-1}\right)=z$
					
					$\therefore z\left[\frac{1}{1!}+\frac{1}{2!}z+...\cdots+\frac{1}{(n+1)!}z^n+R_1(z)\right]\left[B_0+B_1 z+\cdots+\frac{B_n}{n!}z^n+R_2(z)\right]=z$
					
					对比常数项可知:$B_0=1$
					进一步,如果n大于1,对比系数可得:
					$B_0\cdot\frac{1}{(n+1)!}+B_1\cdot\frac{1}{n!}+\cdots+B_k\cdot\frac{1}{k!}\cdot\frac{1}{(n-k)!}\cdots\frac{B_n}{n!}\cdot\frac{1}{1!}=0$
					
					左右同乘$(n+1)!$,得:
					$B_0\cdot\frac{(n+1)!}{(n+1)!\cdot 0!}+B_1\cdot\frac{(n+1)!}{n!\cdot 1!}+\cdots+B_k\frac{(n+1)!}{k!\cdot (n-k)!}+\cdots+B_n\cdot\frac{(n+1)!}{n!\cdot1!}=0$
					
					$\Rightarrow \sum\limits_{k=0}^{n} \binom{n+1}{k}B_k=0$
					
					$\Rightarrow \sum\limits_{k=0}^{n-1} \binom{n+1}{k}B_k=-(n+1)B_n$
					
					$\Rightarrow B_n=-\frac{1}{n+1}\sum\limits_{k=0}^{n-1}\binom{n+1}{k}B_{k}$
					
				\end{proof}
				\point{性质}
				\begin{para}{1}
					\point{除1外,所有奇数项的伯努利数为0}
					\begin{proposition}
						$B_{2n+1}=\begin{cases}
							0,n\geqslant 1 \\
							-\frac{1}{2},n=0 
						\end{cases}$
					\end{proposition}
					\begin{proof}
						由前面的伯努利数定义可知,$f(z)=\frac{z}{e^z-1}$在$\C-\{0\}$上解析
						
						考虑函数$\phi(x)=\frac{f(z)}{z^{2n+2}}$的洛朗展开
						
						$\phi(z)=\frac{B_0}{0!}z^{-2n-2}+\cdots+\frac{B_{2n+1}}{(2n+1)!}z^{-1}+\cdots$
						
						$\Rightarrow Res_{z=0}\phi(z)=\frac{B_{2n+1}}{(2n+1)!}$
						
						$\Rightarrow \oint_{C} \phi(z) \d z = 2\pi i\cdot \frac{B_{2n+1}}{(2n+1)!}$,其中$C=\{|z|=1\}$
						
						记上述积分为$I$,做代换$z=e^{i\theta},\theta\in[0,2\pi]$
						
						$I=\int_{0}^{2\pi} i e^{i\theta}\frac{e^{i\theta}}{e^{e^{i\theta}}-1}e^{-(2n+2)i\theta} \d \theta$
						
						$=\int_{0}^{2\pi} \frac{e^{-2ni\theta}}{e^{e^{i\theta}}-1}\d\theta$
						
						$=\int_{0}^{\pi} \frac{e^{-2ni\theta}}{e^{e^{i\theta}}-1}\d\theta+\int_{\pi}^{2\pi} \frac{e^{-2ni\theta}}{e^{e^{i\theta}}-1}\d\theta$
						
						$=\int_{0}^{\pi} \frac{e^{-2ni\theta}}{e^{e^{i\theta}}-1}\d\theta+\int_{\pi}^{2\pi} \frac{e^{-2ni\theta+2n\pi i}}{e^{e^{i\theta}}-1}\d\theta$
						
						作代换$\alpha = \theta-\pi$,并利用标准指数函数在虚轴的半个周期上变号,得到:
						
						$=\int_{0}^{\pi} \frac{e^{-2ni\theta}}{e^{e^{i\theta}}-1}\d\theta+\int_{0}^{\pi} \frac{e^{-2ni\alpha+2n\pi i}}{e^{-e^{i\alpha}}-1}\d\alpha$
						
						$=\int_{0}^{\pi} -e^{-2ni\theta}=\begin{cases}
							0,n \geqslant 0 \\
							-\pi,n=0
						\end{cases}$
						
						$\Rightarrow B_{2n+1}=\begin{cases}
							0,n\geqslant 1 \\
							-\frac{1}{2},n=0 
						\end{cases}$
					\end{proof}
					\point{}
					\begin{proposition}
						$B_n=-n\zeta(1-n),n\geqslant 1$
					\end{proposition}
					
					\begin{proof}
						黎曼Zeta函数的定义为;
						
						$\zeta(s)=\begin{cases}
							\sum\limits_{k=1}^{\infty}\frac{1}{k^s},Re(s)>1 \\
							\frac{1}{1-2^{1-s}}\sum\limits_{k=1}^{\infty}\frac{(-1)^{k+1}}{k^s},0< Re(s) \leqslant 1 \\
							2^s \pi^{s-1} \sin\left(\frac{\pi s}{2}\right)\Gamma(1-s)\zeta(1-s),Re(s)\leqslant 0
						\end{cases}$
						
						解析延拓过程可以参考《复分析笔记》
						
						先考虑$n=2k+1,k>0$的情形,此时依Zeta函数定义,$\zeta(-2k)=0$,因为$\sin(n\pi)=0$,命题成立。
						
						而当$n=1$,$\zeta(0)=-\frac{1}{2},B_1=-\frac{1}{2}$,命题成立。
						
						接下来考虑n为偶数的情形:
						
						我们欲证$B_{2n} = -2n\zeta(1-2n)$
						
						$\Rightarrow B_{2n} = -2n 2^{1-2n} \pi^{-2n} \sin\left(\frac{\pi (1-2n)}{2}\right)\Gamma(2n)\zeta(2n)$
						
						考虑函数$f_n(z) = \frac{z^{-2n}}{e^z -1}$
						
						$=z^{-2n-1}\frac{z}{e^z-1}$
						
						$=z^{-2n-1}\sum\limits_{k=0}^{\infty} \frac{B_k}{k!}x^k$
						
						$=\sum\limits_{k=0}^{\infty} \frac{B_k}{k!}x^{k-2n-1}$
						
						对比-1阶系数,得:
						
						$Res_{z=0} f_n(z) = \frac{B_{2n}}{(2n)!}$
						
						再考虑$z=2\pi ki$处的留数
						
						$Res_{z=2\pi ki} f_n(z) = \lim\limits_{z\to 2\pi ni} \frac{z^{-2n}}{e^z-1}(z-2\pi ki)$
						
						$=(2\pi ki)^{-2n}=(-1)^n (2\pi k)^{-2n}$
						
						记$C_N = \{|z|=(2N+1)\pi\}$,$G_N$为$C_0,C_N$围成的区域
						
						$\therefore \int_{\partial G_N} f_n(z)=\int_{C_N} f_n(z)-\int_{C_0} f_n(z)$
						
						而$\int_{\partial G_N} = 2\pi i \sum\limits_{k=-N}^{N} Res_{z=2\pi ki} f_n(z)$
						
						$=2\pi i \sum\limits_{k=-N}^{N} (-1)^n (2\pi k)^{-2n}$
						
						$=4\pi i \sum\limits_{k=1}^{N} (-1)^n (2\pi k)^{-2n}$
						
						接下来考察$\int_{C_N} f_n(z)$,因为在$C_N$上$|\frac{1}{e^z-1}|$有界,而:
						
						$\int_{C_N} z^{-2n} = \int_{C_N} i e^{i\theta} N^{-2n} = O(N^{-2n})$
						
						所以$\int_{C_N} f_n(z) = O(N^{-2n})$
						
						再接下来考察$\int_{C_0} f_n(z)$
						
						$\int_{C_0} f_n(z) = 2\pi i Res_{z=0} f_n(z) = \frac{2\pi i B_{2n}}{(2n)!}$
						
						$\because \int_{\partial G_N} f_n(z)=\int_{C_N} f_n(z)-\int_{C_0} f_n(z)$
						
						$\therefore 4\pi i \sum\limits_{k=1}^{N} (-1)^n (2\pi k )^{-2n} = O(N^{-2n})-2\pi i\frac{2\pi i B_{2n}}{(2n)!}$
						
						取$N\to +\infty$,那么$\sum\limits_{k=1}^{\infty} 2(-1)^n (2\pi k)^{-2n}=-\frac{B_{2n}}{(2n)!}$
						
						$\Rightarrow B_{2n} = (2n)!\sum\limits_{k=1}^{\infty} 2(-1)^{n+1} (2\pi k)^{-2n}$
						
						$\Rightarrow B_{2n} = (2n)!\frac{2(-1)^{n+1}}{(2\pi)^{2n}}\sum\limits_{k=1}^{\infty} \frac{1}{k^{2n}}$
						
						$\Rightarrow B_{2n} = (2n)!\frac{2(-1)^{n+1}}{(2\pi)^{2n}}\zeta(2n)$
						
						$\Rightarrow B_{2n} = 2n (-1)^{n+1} 2^{1-2n} \pi^{-2n} \Gamma(2n)\zeta(2n)$
						
						只需注意到,$(-1)^{n+1}=-\sin\left(\frac{(1-2n)\pi}{2}\right)$,那么这一结果就是我们所证的命题。
					\end{proof}
					这一证明的关键在于注意到$\frac{z^{-2n}}{e^z-1}$在$0$处的留数与伯努利数有关,而其他奇点处的留数之和与自然数负幂求和有关,从而联系了伯努利数和Zeta函数。
					\point{伯努利数偶数项相互交错}
						\begin{proposition}
							$B_{4k} < 0,B_{4k+2} > 0,k \geqslant 1$
						\end{proposition}
						\begin{proof}
							我们利用之前命题中的结果$B_{2n} = (2n)!\frac{2(-1)^{n+1}}{(2\pi)^{2n}}\zeta(2n)$
							
							当n分别为$2k,2k+1$时:
							
							$B_{4k} = (4k)!\frac{2(-1)^{2k+1}}{(2\pi)^{4k}}\zeta(4k)=-(4k)!\frac{2}{(2\pi)^{4k}}\zeta(4k)$
							
							$B_{4k+2} = (4k+2)!\frac{2(-1)^{2k+2}}{(2\pi)^{4k+2}}\zeta(4k+2)=(4k+2)!\frac{2}{(2\pi)^{4k+2}}\zeta(4k+2)$
							
							而Zeta函数对于全体正整数取正值,命题得证。
						\end{proof}
					\point{从$B_6$开始,伯努利数的绝对值依次递增}
						\begin{proposition}
							$|B_{2k}|<|B_{2k+2}|,k \geqslant 3$
						\end{proposition}
						\begin{proof}
							在之前的命题中,我们证明了:$\Rightarrow B_{2n} = (2n)!\frac{2(-1)^{n+1}}{(2\pi)^{2n}}\zeta(2n)$
							
							$\therefore |B_{2k}| = (2k)!\frac{2}{(2\pi)^{2k}}\zeta(2k)$
							
							$|B_{2k+2}| = (2k+2)!\frac{2}{(2\pi)^{2k+2}}\zeta(2k+2)$
							
							两者作差,$|B_{2k+2}|-|B_{2k}|$
							
							$=2\frac{(2k+2)!\zeta(2k+2)-(2k)!\zeta(2k)(2\pi)^2}{(2\pi)^{2k+2}}$
							
							因为当$s>1,s \in \N_+$时,$\zeta(s)$单调递减且$\lim\limits_{s\to\infty}\zeta(s) = 1$
							
							我们计算得$\zeta(6)=\frac{\pi^2}{945}$,所以可以作估计:$\zeta(2k+2)>1,\zeta(2k)<\frac{\pi^2}{945}$
							
							$\therefore (2k+2)!\zeta(2k+2)-(2k)!\zeta(2k)(2\pi)^2$
							
							$> (2k+2)!-(2k)!\cdot 4\pi^2 \frac{\pi^6}{945}$
							
							$=\left((2k+2)(2k+1)- 4\pi^2 \frac{\pi^6}{945}\right)(2k)!>0$
							
						\end{proof}
					\point{自然幂指数和的通式}
					\begin{proposition}
						$S_p(n)=\sum\limits_{k=1}^{n}k^p=\frac{1}{p+1}\sum\limits_{i=0}^{p} (-1)^i B_i \binom{p+1}{i} n^{p+1-i}$
					\end{proposition}
					\begin{proof}
						明天再说,累了
					\end{proof}
				\end{para}
			\end{para}
		\subsection{欧拉数}
	% 附录:欧拉-麦克劳林公式
	\section{欧拉-麦克劳林公式}
		我们直观上会认为,一个连续的级数和应该可以用这一整数区间的积分来大致模拟。一些结果也的确呈现了这个特征,比如以下有关调和级数的命题:
		
		$\sum\limits_{n=1}^{x} \frac{1}{n} = \int_{1}^{x} \frac{1}{t}\d t +o(1)$
		
		特别地,其中的余项有$o(1)\to\varepsilon$,$\varepsilon$为欧拉常数.
		
		我们对这里的余项做探讨,探究何时可以积分和求和的差值为常数阶,以及这一余项大小如何。
		\subsection{伯努利多项式}
		\begin{para}{0}
			\point{定义}
			\begin{defn}{伯努利多项式的级数定义}{}
				满足以下级数的函数族$B_n(x)$称为伯努利多项式
				\begin{equation}
					\frac{z e^{xz}}{e^z-1}=\sum\limits_{n=0}^{\infty} \frac{B_n(x)}{n!}z^n
				\end{equation}
			\end{defn}
			\point{性质}
			\begin{para}{1}
				\point{}
					\begin{proposition}
						$B_n(0)=B_n,B_n(1)=(-1)^{n}B_n$
					\end{proposition}
					\begin{proof}
						 在伯努利多项式的定义中分别代入$0,1$:
						 
						 $\frac{z}{e^z-1}=\sum\limits_{n=0}^{\infty} \frac{B_n(0)}{n!}$
						 
						 对比伯努利数的定义即得$B_n(0) = B_n$
						 
						 $\frac{z e^z}{e^z-1}=\sum\limits_{n=0}^{\infty} \frac{B_n(1)}{n!}z^n$
						 
						 $\Rightarrow\frac{z}{1-e^{-z}}=\frac{-z}{e^{-z}-1} = \sum\limits_{n=0}^{\infty} \frac{B_n}{n!}z^n$
					\end{proof}
				\point{}
					\begin{proposition}
						$B_n(x) = \sum\limits_{i=0}^{n} \binom{n}{i}B_i x^{n-i}$
					\end{proposition}
					\begin{proof}
						依伯努利数定义,$\frac{z e^{xz}}{e^z-1}=\sum\limits_{k=0}^{\infty} \frac{B_k(x)}{k!}z^k$
						
						那么$\frac{z e^{xz}}{e^z-1}=\frac{z}{e^z-1} e^{xz}$
						
						$=\sum\limits_{i=0}^{\infty} B_i \frac{z^i}{i!} \cdot \sum\limits_{j=0}^{\infty} \frac{(xz)^j}{j!}$
						
						$=\sum\limits_{k=0}^{\infty} \left(\sum\limits_{i=0}^{\infty} B_i \frac{x^i}{i!}\cdot \frac{(xz)^{k-i}}{(k-i)!}\right)$
						
						$=\sum\limits_{k=0}^{\infty} \left(\sum\limits_{i=0}^{\infty} \frac{B_i x^{k-i}}{(k-i)!i!}z^k\right)$
						
						$=\sum\limits_{k=0}^{\infty} \left(\sum\limits_{i=0}^{\infty} \frac{B_i k!}{(k-i)!i!}x^{k-i}\right) z^k$
						
						比较系数即得:$B_n(x) = \sum\limits_{i=0}^{n} \binom{n}{i}B_i x^{n-i}$。
					\end{proof}
				\point{}
					\begin{proposition}
						$B_n^{\prime} (x) = n B_{n-1} (x)$
					\end{proposition}
					\begin{proof}
						依伯努利数定义:$\frac{z e^{xz}}{e^z-1} = \sum\limits_{n=0}^{\infty} \frac{B_n(x)}{n!}z^n$
						
						对两侧求对x的导数得:$\frac{z^2 e^{xz}}{e^z-1} = \sum\limits_{n=1}^{\infty} \frac{B_n^{\prime}(x)}{n!}z^n$
						
						此处把求和下限调整为1是因为注意到,级数在$i=0$时为常数,因此此项导数为0。
						
						两侧同除z得:$\frac{z e^{xz}}{e^z-1} = \sum\limits_{n=1}^{\infty} \frac{B_n^{\prime}(x)}{n!}z^{n-1}$
						
						$=\sum\limits_{n=0}^{\infty} \frac{B_{n+1}^{\prime}(x)}{(n+1)!}z^{n}$
						
						$=\sum\limits_{n=0}^{\infty} \frac{B_{n+1}^{\prime}(x)}{(n+1)\cdot n!}z^{n}$
						
						对比系数得:$B_n(x) = \frac{B_{n+1}^{\prime}(x)}{n+1}$,于是命题得证。
					\end{proof}
				\point{}
					\begin{proposition}
						$\int_{0}^{1} B_n (x) \d x = 0 ,n \geqslant 1$
					\end{proposition}
					\begin{proof}
						利用上面给出的导数性质:
						
						$\int_{0}^{1} B_n (x) \d x = \frac{B_{n+1}(x)}{n+1} \bigg|_{0}^{1} = \frac{1}{n+1}\left(B_{n+1}(1)-B_n(0)\right)$
						
						那么只需要注意到,当$n\geqslant 2$,如果$n$为偶,按前面的断言有$B_n(1)=B_n=B_n(0)$;而如果$n$为奇,那么$B_n(1) = B_n(0) = B_n = 0$,于是命题得证。
					\end{proof}
				\point{}
					\begin{proposition}
						$B_n (x+1)-B_n (x) = nx^{n-1}$
					\end{proposition}
					\begin{proof}
						考虑构造与伯努利多项式生成函数类似的级数:
						
						$\sum\limits_{n=0}^{\infty} \left(B_n(x+1)-B_n(x)\right) \frac{x^n}{n!}$
						
						$=\sum\limits_{n=0}^{\infty} B_n(x+1) \frac{x^n}{n!}-\sum\limits_{n=0}^{\infty} B_n(x) \frac{x^n}{n!}$
						
						$=\frac{z e^{(x+1)z}}{e^z-1}-\frac{z e^{xz}}{e^z-1}$
						
						$=\frac{z e^{xz} (e^z-1)}{e^z-1} = z e^{xz}$
						
						$=\sum\limits_{n=0}^{\infty} x^n \frac{z^{n+1}}{n!}$
						
						$=\sum\limits_{n=0}^{\infty} (n+1)x^n \frac{z^{n+1}}{(n+1)!}$
						
						$=\sum\limits_{n=1}^{\infty} n x^n \frac{z^{n}}{(n)!}$
						
						$=\sum\limits_{n=0}^{\infty} n x^n \frac{z^{n}}{(n)!}$,因为此级数的$n=0$项为0。
						
						于是对比系数即得证。
					\end{proof}
				\point{}
					\begin{proposition}
						$B_n (1-x) = (-1)^n B_n(x)$
					\end{proposition}
					\begin{proof}
						考虑以下级数:
						
						$\sum\limits_{n=0}^{\infty} (-1)^n B_n(x) \frac{z^n}{n!}$
						
						$=\sum\limits_{n=0}^{\infty} B_n(x) \frac{(-z)^n}{n!}$
						
						$=\frac{-z e^{-xz}}{e^{-z}-1}$
						
						上下同乘$-e^z$,$=\frac{z e^{(1-x)z}}{e^z-1}$
						
						$=\sum\limits_{n=0}^{\infty} B_n(1-x) \frac{z^n}{n!}$,于是命题得证
					\end{proof}
			\end{para}
		\end{para}
		\begin{them}{欧拉-麦克劳林公式}{}
			假设$f(x)$无穷阶可导,那么:
			\begin{equation}
				\sum\limits_{a\leq n<b}f(n) = \int_{a}^{b}f(x)\d x+\sum\limits_{k=1}^{\infty}\frac{B_k}{k!}f^{(k-1)}(x)\big|_{a}^{b}+{(-1)}^{m}\int_{a}^{b}\frac{B_m({x})}{m!}f^{(m)}(x)\d x
			\end{equation}
			
		\end{them}
	
	%附录:其他四种三角函数的泰勒展开
	\section{其他四种三角函数的泰勒/洛朗展开}
		\subsection{$\cot x$}
			借助伯努利数或者Gamma函数,分别可以将余切分解为两种形式
			\begin{them}{余切函数的洛朗展开}{}
				\begin{equation}
					\cot x = \sum\limits_{n=0}^{\infty}\frac{2(-1)^n B_{2n} (2x)^{2n-1}}{(2n)!}
				\end{equation}
				这一展开是$\cot x$的洛朗展开而非泰勒展开,因为$\cot x \sim \frac{1}{x},x\to 0$,在0处不解析
			\end{them}
			\begin{proof}
				依余切函数定义:
				$\cot x = \frac{i(e^{ix}+e^{-ix})}{e^{ix}-e^{-ix}}$
				
				$=i\frac{e^{2ix}+1}{e^{2ix}-1}$
				
				$=i\left(1+\frac{2}{e^{2ix}-1}\right)$
				再利用伯努利数的级数定义:
				
				$\cot x = i+2i\cdot \frac{1}{2ix} \sum\limits_{n=0}^{\infty} \frac{B_n (2ix)^n}{n!}$
				
				$\cot x = i+2i\sum\limits_{n=0}^{\infty} \frac{B_n (2ix)^{n-1}}{n!}$
				因为伯努利数的奇数项只有$B_1=-\frac{1}{2}$,其余全为0,故:
				
				$\cot x = i+2i\left(\sum_{n=0}^{\infty} \frac{B_2n (2ix)^2n-1}{2n!} -\frac{1}{2}\right)$
				
				$=\sum\limits_{n=0}^{\infty} \frac{2 B_{2n} (i)^{2n} (2x)^{2n-1}}{2n!}$
				$=\sum\limits_{n=0}^{\infty}\frac{2(-1)^n B_{2n} (2x)^{2n-1}}{(2n)!}$
			\end{proof}
			\begin{them}{余切函数的第二种级数展开}{}
				\begin{equation}
					\cot z = \frac{1}{z}+\sum\limits_{n=0}^{\infty} \left(\frac{1}{z-n\pi}+\frac{1}{z+n\pi}\right)
				\end{equation}
			\end{them}
			\begin{proof}
				利用正弦函数的无穷乘积展开式:
				
				$\sin \pi z = \pi z \prod\limits_{n=1}^{\infty} \left(1-\frac{z^2}{n^2}\right)$
				
				将$\pi z$移至左侧并取对数,得:
				
				$\ln \frac{\sin \pi z}{\pi z} = \sum\limits_{n=1}^{\infty} \ln \left(1-\frac{z^2}{n^2}\right)$
				
				双侧求导,得:
				
				$\frac{\pi z}{\sin \pi z}\left(\frac{\cos \pi z}{z}-\frac{\sin \pi z}{\pi z^2}\right) = \sum\limits_{n=1}^{\infty} \frac{-2z}{n^2 (1-\frac{z^2}{n^2})}$
				
				进行一些基本运算并对级数内部裂项,得:
				
				$\pi \cot \pi z = \frac{1}{z}+\sum\limits_{n=1}^{\infty}\left(\frac{1}{z-n}+\frac{1}{z+n}\right)$
				
				$\Rightarrow \cot \pi z = \frac{1}{\pi z}+\sum\limits_{n=1}^{\infty}\left(\frac{1}{\pi z-\pi n}+\frac{1}{\pi z+\pi n}\right)$
				
				$\Rightarrow \cot z = \frac{1}{z}+\sum\limits_{n=0}^{\infty} \left(\frac{1}{z-n\pi}+\frac{1}{z+n\pi}\right)$
			\end{proof}
			这一证明的注意力很强,其整体思路是:将$\cot$转换为$\cos$和$\sin$的除法,再利用正弦函数的导数,构造相应的对数除式,而恰好注意到,$\sin z$的无穷乘积展开正适合构造对数求和。
		\subsection{$\tan x$}
			\begin{them}{正切函数的Taylor展开}{}
				\begin{equation}
					\tan x = \sum\limits_{n=1}^{\infty} \frac{(-1)^{n-1}2^{2n}(2^{2n}-1)B_{2n}}{(2n)!}x^{2n-1}
				\end{equation}
			\end{them}
			\begin{proof}
				注意到:$\cot x = -2 \cot 2x +\cot x$
				
				因为:$-2\cot 2x+\cot x = -2\frac{\cos 2x}{\sin 2x}+\frac{\cos x}{\sin x}$
				
				$=-\frac{\cos^2 x-\sin^2 x}{\sin x\cos x}+\frac{\cos x}{\sin x}$
				
				$=\frac{\sin^2 x-\cos^2 x}{\sin x\cos x}+\frac{\cos^2 x}{\sin x\cos x}$
				
				$=\frac{\sin^2 x}{\sin x\cos x}$
				
				$=\frac{\sin x}{\cos x}=\cot x$
				
				$\therefore \cot x = -2\sum\limits_{n=0}^{\infty}\frac{2(-1)^n B_{2n} (4x)^{2n-1}}{(2n)!}+\sum\limits_{n=0}^{\infty}\frac{2(-1)^n B_{2n} (2x)^{2n-1}}{(2n)!} $
				
				$=-\sum\limits_{n=0}^{\infty}\frac{2(-1)^n B_{2n} (1-2^{2n}) (2x)^{2n-1}}{(2n)!}$
				
				$=\sum\limits_{n=0}^{\infty}\frac{(-1)^{n-1} B_{2n} 2^{2n}(2^{2n}-1) x^{2n-1}}{(2n)!}$
				
				又当$n=0$时一般项为0,
				
				$\therefore \cot x = \sum\limits_{n=1}^{\infty}\frac{(-1)^{n-1} 2^{2n}(2^{2n}-1) B_{2n}}{(2n)!}  x^{2n-1}$
			\end{proof}
			\begin{proof}
				怎么开始套娃了(恼)
			\end{proof}
	
	%附录:不定积分初等性判定
	\section{原函数初等性的判定方法}
		\subsection{切比雪夫定理}
			\begin{them}{切比雪夫定理}{}
				设$m,n,p\in \Q-\{0\}$,那么以下积分
				\begin{equation}
					\int x^m(a+bx^n)^p \d x
				\end{equation}
				初等的充要条件是:$p,\frac{m+1}{n},\frac{m+1}{n}+p$中至少有一个为整数
			\end{them}
		\subsection{刘维尔定理}
			在介绍刘维尔定理前,需要先介绍一些微分代数的概念:
			
			首先我们扩展微分的概念。我们将满足类似乘法、除法微分性质的泛函也称为微分。
			
			先引入微分域及其常数域
			\begin{para}{0}
				\point{微分域}
					\begin{defn}{微分域}{}
						一个由函数组成的域$F$及其上的一个算子$\delta:F \rightarrow F$,如果$\forall f,g \in F$有:
						
						$\ding{172} \delta(f+g) = \delta(f)+\delta(g)$
						
						$\ding{173} \delta(fg) = \delta(f)\cdot g+f \cdot \delta(g)$
						
						那么称$(F,\delta)$是一个微分域
					\end{defn}
					容易验证$\delta$是线性算子,于是我们有时简记$\delta(f)$为$\delta f$
					\begin{defn}{微分域的常数域}{}
						微分域$(F,\delta)$的常数域定义为:
						
						$Con (F,\delta) =\{f \in F| \delta f = 0\}$
					\end{defn}
					同时定义域的扩张:
					\begin{defn}{域的扩张}{}
						设$F,K$是两个域,并且$K$是满足$F \subseteq K$且包含$h \subseteq K$的最小域(即$K$是任何满足上述条件的域的子域),记作$K = F(h)$
					\end{defn}
					作为接下来内容的预备,我们先验证那些显然的微分性质:
					\begin{proposition}
						$\delta C = 0$,其中$C$为常数 
					\end{proposition}
					\begin{proof}
						只需要验证$\delta 1 = 0$
						
						那么有:$\delta (1\cdot 1) = \delta 1 \cdot 1 + 1 \cdot \delta 1=2\delta 1$
						
						于是有$\delta 1 = 0$,利用微分的线性即得证。
					\end{proof}
					\begin{proposition}
						$\delta \left(\frac{f}{g}\right)=\frac{\delta f \cdot g - f \delta g}{g^2}$
					\end{proposition}
					\begin{proof}
						首先推导$\delta \left(\frac{1}{g}\right)$
						
						$\because \delta 1 = \delta \left(g \cdot \frac{1}{g}\right) = 0$
						
						$\Rightarrow \delta g \frac{1}{g}+g \delta \left(\frac{1}{g}\right)=0$
						
						$\Rightarrow \delta \left(\frac{1}{g}\right) = -\frac{\delta g}{g^2}$
						
						于是$\delta \left(\frac{f}{g}\right) = \delta \left(f \cdot \frac{1}{g}\right)$
						
						$=\delta f \frac{1}{g}-f \frac{\delta g}{g^2} = \frac{\delta f \cdot g - f \delta g}{g^2}$
					\end{proof}
				\point{微分域的初等扩张}
					接下来讨论什么是“初等”的函数。
					\begin{defn}{微分域的初等扩张}{}
						设$(F,\delta),(K,\delta)$是两个微分域,$h \in K$并且$K = F(h)$,那么:
						
						$\ding{172}$如果存在$F$中的一个多项式$p(x) \in F[x]$,有$p(h)=0$,那么称$h$是$F$的一个代数元素,$K=F(h)$是$F$的单代数扩张
						
						$\ding{173}$如果存在$F$中的一个函数$f$,使得$\delta h = \frac{\delta f}{f}$,那么称$K=F(h)$是$F$的单对数扩张
						
						$\ding{173}$如果存在$F$中的一个函数$f$,使得$\frac{\delta h}{h} = \delta f$,那么称$K=F(h)$是$F$的单指数扩张。
						
						单对数扩张和单指数扩张统称为单超越扩张,其对应的$h$称为$F$的超越元素;以上三种扩张统称为单初等扩张
						
						有限次初等扩张的复合称为初等扩张
					\end{defn}
					我们也可以在此以另外的方式定义出初等函数:
					\begin{defn}{初等函数}{}
						如果函数$f$处于微分域$\left(C(x),\frac{\d}{\d x}\right)$的某个初等扩张中,那么称$f$是一个初等函数
					\end{defn}
					接下来就可以给出刘维尔定理了。
				\point{刘维尔定理}
					\begin{them}{刘维尔定理}
						设$(F,\delta),(K,\delta)$是两个微分域,$K$是$F$的初等扩张,并且$Con(F,\delta)=Con(K,\delta)$,且$\forall f \in F,\exists g \in K,s.t.\delta g = f$
						
						那么一定$\exists c_1,\cdots,c_n \in Con(F,\delta),u_1,\cdots,u_n,v \in F$,使得
						\begin{equation}
							g = \sum\limits_{i=1}^{n} c_i \ln (u_i)+v
						\end{equation}
					\end{them}
			\end{para}
			
			
	%附录:超越积分的特殊解法
	\section{一些超越积分的特殊解法}
		\subsection{Direchlet积分}
		
	
	\ifx\allfiles\undefined
\end{document}
\fi