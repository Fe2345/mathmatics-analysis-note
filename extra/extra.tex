\ifx\allfiles\undefined
\documentclass[12pt, a4paper, oneside, UTF8]{ctexbook}
\setCJKmainfont{SimSun}
\def\path{../config}
\input{../config/_config}
\begin{document}
	% \input{../config/cover}
	\else
	\fi
	%标题
	\chapter{附录}
	这一部分中,对于正文中因为逻辑结构无法提及的部分,进行补充。包括特殊函数,有趣的数学概念,一些命题的全新解法,以及难以推导的公式
	证明可能使用复分析、实分析、泛函等超纲内容
	%--------------------正文---------------------------
	
	%附录:伯努利数和欧拉数
	\section{伯努利数与欧拉数}
		在对泰勒公式的研究中,我们推到了标准指数函数$\exp x$的级数展开,它的系数是极其容易推导的。
		
		但是,如果我们考察其倒数,那么就变得相当难以推导:因为它的高阶导数一阶比一阶难以求得。
		
		这一附录中我们反其道而行之,先假设系数,再利用其与指数函数乘积求出系数,从而提出了两个重要的数列:伯努利数和欧拉数。
		
		特别地,由于三角函数的分析定义正是利用的指数函数,利用伯努利数和欧拉数可以求出很多之前难以求出的展开式。
		
		\subsection{伯努利数}
			\begin{para}{0}
				\point{定义}
				\begin{defn}{伯努利数的级数定义}{}
					符合以下级数展开的数列称为伯努利数:
					\begin{equation}
						\frac{z}{e^z -1} = \sum_{n=0}^{\infty}\frac{B_n}{n!}z^n
					\end{equation}
				\end{defn}
				以上的定义十分简洁,但是难以计算。但是显然,借助Taylor级数我们可以推导出等价的递归定义:
				\begin{defn}{伯努利数的递归定义}{}
					\begin{equation}
						B_0=1,B_n=-\frac{1}{n+1}\sum\limits_{k=0}^{n-1}\binom{n+1}{k}B_{k}
					\end{equation}
				\end{defn}
				\begin{proof}
					利用级数定义中的展开式反推:
					
					$\because\left(e^z-1\right)\left(\frac{z}{e^z-1}\right)=z$
					
					$\therefore z\left[\frac{1}{1!}+\frac{1}{2!}z+...\cdots+\frac{1}{(n+1)!}z^n+R_1(z)\right]\left[B_0+B_1 z+\cdots+\frac{B_n}{n!}z^n+R_2(z)\right]=z$
					
					对比常数项可知:$B_0=1$
					进一步,如果n大于1,对比系数可得:
					$B_0\cdot\frac{1}{(n+1)!}+B_1\cdot\frac{1}{n!}+\cdots+B_k\cdot\frac{1}{k!}\cdot\frac{1}{(n-k)!}\cdots\frac{B_n}{n!}\cdot\frac{1}{1!}=0$
					
					左右同乘$(n+1)!$,得:
					$B_0\cdot\frac{(n+1)!}{(n+1)!\cdot 0!}+B_1\cdot\frac{(n+1)!}{n!\cdot 1!}+\cdots+B_k\frac{(n+1)!}{k!\cdot (n-k)!}+\cdots+B_n\cdot\frac{(n+1)!}{n!\cdot1!}=0$
					
					$\Rightarrow \sum\limits_{k=0}^{n} \binom{n+1}{k}B_k=0$
					
					$\Rightarrow \sum\limits_{k=0}^{n-1} \binom{n+1}{k}B_k=-(n+1)B_n$
					
					$\Rightarrow B_n=-\frac{1}{n+1}\sum\limits_{k=0}^{n-1}\binom{n+1}{k}B_{k}$
					
				\end{proof}
				\point{性质}
				\begin{para}{1}
					\point{除1外,所有奇数项的伯努利数为0}
					\begin{proposition}
						$B_{2n+1}=\begin{cases}
							0,n\geqslant 1 \\
							-\frac{1}{2},n=0 
						\end{cases}$
					\end{proposition}
					\begin{proof}
						由前面的伯努利数定义可知,$f(z)=\frac{z}{e^z-1}$在$\C-\{0\}$上解析
						
						考虑函数$\phi(x)=\frac{f(z)}{z^{2n+2}}$的洛朗展开
						
						$\phi(z)=\frac{B_0}{0!}z^{-2n-2}+\cdots+\frac{B_{2n+1}}{(2n+1)!}z^{-1}+\cdots$
						
						$\Rightarrow Res_{z=0}\phi(z)=\frac{B_{2n+1}}{(2n+1)!}$
						
						$\Rightarrow \oint_{C} \phi(z) \d z = 2\pi i\cdot \frac{B_{2n+1}}{(2n+1)!}$,其中$C=\{|z|=1\}$
						
						记上述积分为$I$,做代换$z=e^{i\theta},\theta\in[0,2\pi]$
						
						$I=\int_{0}^{2\pi} i e^{i\theta}\frac{e^{i\theta}}{e^{e^{i\theta}}-1}e^{-(2n+2)i\theta} \d \theta$
						
						$=\int_{0}^{2\pi} \frac{e^{-2ni\theta}}{e^{e^{i\theta}}-1}\d\theta$
						
						$=\int_{0}^{\pi} \frac{e^{-2ni\theta}}{e^{e^{i\theta}}-1}\d\theta+\int_{\pi}^{2\pi} \frac{e^{-2ni\theta}}{e^{e^{i\theta}}-1}\d\theta$
						
						$=\int_{0}^{\pi} \frac{e^{-2ni\theta}}{e^{e^{i\theta}}-1}\d\theta+\int_{\pi}^{2\pi} \frac{e^{-2ni\theta+2n\pi i}}{e^{e^{i\theta}}-1}\d\theta$
						
						作代换$\alpha = \theta-\pi$,并利用标准指数函数在虚轴的半个周期上变号,得到:
						
						$=\int_{0}^{\pi} \frac{e^{-2ni\theta}}{e^{e^{i\theta}}-1}\d\theta+\int_{0}^{\pi} \frac{e^{-2ni\alpha+2n\pi i}}{e^{-e^{i\alpha}}-1}\d\alpha$
						
						$=\int_{0}^{\pi} -e^{-2ni\theta}=\begin{cases}
							0,n \geqslant 0 \\
							-\pi,n=0
						\end{cases}$
						
						$\Rightarrow B_{2n+1}=\begin{cases}
							0,n\geqslant 1 \\
							-\frac{1}{2},n=0 
						\end{cases}$
					\end{proof}
					\point{}
					\begin{proposition}
						$B_n=-n\zeta(1-n),n\geqslant 1$
					\end{proposition}
					
					\begin{proof}
						黎曼Zeta函数的定义为;
						
						$\zeta(s)=\begin{cases}
							\sum\limits_{k=1}^{\infty}\frac{1}{k^s},Re(s)>1 \\
							\frac{1}{1-2^{1-s}}\sum\limits_{k=1}^{\infty}\frac{(-1)^{k+1}}{k^s},0< Re(s) \leqslant 1 \\
							2^s \pi^{s-1} \sin\left(\frac{\pi s}{2}\right)\Gamma(1-s)\zeta(1-s),Re(s)\leqslant 0
						\end{cases}$
						
						解析延拓过程可以参考《复分析笔记》
						
						先考虑$n=2k+1,k>0$的情形,此时依Zeta函数定义,$\zeta(-2k)=0$,因为$\sin(n\pi)=0$,命题成立。
						
						而当$n=1$,$\zeta(0)=-\frac{1}{2},B_1=-\frac{1}{2}$,命题成立。
						
						接下来考虑n为偶数的情形:
						
						我们欲证$B_{2n} = -2n\zeta(1-2n)$
						
						$\Rightarrow B_{2n} = -2n 2^{1-2n} \pi^{-2n} \sin\left(\frac{\pi (1-2n)}{2}\right)\Gamma(2n)\zeta(2n)$
						
						考虑函数$f_n(z) = \frac{z^{-2n}}{e^z -1}$
						
						$=z^{-2n-1}\frac{z}{e^z-1}$
						
						$=z^{-2n-1}\sum\limits_{k=0}^{\infty} \frac{B_k}{k!}x^k$
						
						$=\sum\limits_{k=0}^{\infty} \frac{B_k}{k!}x^{k-2n-1}$
						
						对比-1阶系数,得:
						
						$Res_{z=0} f_n(z) = \frac{B_{2n}}{(2n)!}$
						
						再考虑$z=2\pi ki$处的留数
						
						$Res_{z=2\pi ki} f_n(z) = \lim\limits_{z\to 2\pi ni} \frac{z^{-2n}}{e^z-1}(z-2\pi ki)$
						
						$=(2\pi ki)^{-2n}=(-1)^n (2\pi k)^{-2n}$
						
						记$C_N = \{|z|=(2N+1)\pi\}$,$G_N$为$C_0,C_N$围成的区域
						
						$\therefore \int_{\partial G_N} f_n(z)=\int_{C_N} f_n(z)-\int_{C_0} f_n(z)$
						
						而$\int_{\partial G_N} = 2\pi i \sum\limits_{k=-N}^{N} Res_{z=2\pi ki} f_n(z)$
						
						$=2\pi i \sum\limits_{k=-N}^{N} (-1)^n (2\pi k)^{-2n}$
						
						$=4\pi i \sum\limits_{k=1}^{N} (-1)^n (2\pi k)^{-2n}$
						
						接下来考察$\int_{C_N} f_n(z)$,因为在$C_N$上$|\frac{1}{e^z-1}|$有界,而:
						
						$\int_{C_N} z^{-2n} = \int_{C_N} i e^{i\theta} N^{-2n} = O(N^{-2n})$
						
						所以$\int_{C_N} f_n(z) = O(N^{-2n})$
						
						再接下来考察$\int_{C_0} f_n(z)$
						
						$\int_{C_0} f_n(z) = 2\pi i Res_{z=0} f_n(z) = \frac{2\pi i B_{2n}}{(2n)!}$
						
						$\because \int_{\partial G_N} f_n(z)=\int_{C_N} f_n(z)-\int_{C_0} f_n(z)$
						
						$\therefore 4\pi i \sum\limits_{k=1}^{N} (-1)^n (2\pi k )^{-2n} = O(N^{-2n})-2\pi i\frac{2\pi i B_{2n}}{(2n)!}$
						
						取$N\to +\infty$,那么$\sum\limits_{k=1}^{\infty} 2(-1)^n (2\pi k)^{-2n}=-\frac{B_{2n}}{(2n)!}$
						
						$\Rightarrow B_{2n} = (2n)!\sum\limits_{k=1}^{\infty} 2(-1)^{n+1} (2\pi k)^{-2n}$
						
						$\Rightarrow B_{2n} = (2n)!\frac{2(-1)^{n+1}}{(2\pi)^{2n}}\sum\limits_{k=1}^{\infty} \frac{1}{k^{2n}}$
						
						$\Rightarrow B_{2n} = (2n)!\frac{2(-1)^{n+1}}{(2\pi)^{2n}}\zeta(2n)$
						
						$\Rightarrow B_{2n} = 2n (-1)^{n+1} 2^{1-2n} \pi^{-2n} \Gamma(2n)\zeta(2n)$
						
						只需注意到,$(-1)^{n+1}=-\sin\left(\frac{(1-2n)\pi}{2}\right)$,那么这一结果就是我们所证的命题。
					\end{proof}
					这一证明的关键在于注意到$\frac{z^{-2n}}{e^z-1}$在$0$处的留数与伯努利数有关,而其他奇点处的留数之和与自然数负幂求和有关,从而联系了伯努利数和Zeta函数。
					\point{伯努利数偶数项相互交错}
						\begin{proposition}
							$B_{4k} < 0,B_{4k+2} > 0,k \geqslant 1$
						\end{proposition}
						\begin{proof}
							我们利用之前命题中的结果$B_{2n} = (2n)!\frac{2(-1)^{n+1}}{(2\pi)^{2n}}\zeta(2n)$
							
							当n分别为$2k,2k+1$时:
							
							$B_{4k} = (4k)!\frac{2(-1)^{2k+1}}{(2\pi)^{4k}}\zeta(4k)=-(4k)!\frac{2}{(2\pi)^{4k}}\zeta(4k)$
							
							$B_{4k+2} = (4k+2)!\frac{2(-1)^{2k+2}}{(2\pi)^{4k+2}}\zeta(4k+2)=(4k+2)!\frac{2}{(2\pi)^{4k+2}}\zeta(4k+2)$
							
							而Zeta函数对于全体正整数取正值,命题得证。
						\end{proof}
					\point{自然幂指数和的通式}
					\begin{proposition}
						$S_p(n)=\sum\limits_{k=1}^{n}k^p=\frac{1}{p+1}\sum\limits_{i=0}^{p} (-1)^i B_i \binom{p+1}{i} n^{p+1-i}$
					\end{proposition}
					\begin{proof}
						明天再说,累了
					\end{proof}
				\end{para}
			\end{para}
		\subsection{欧拉数}
	% 附录:欧拉-麦克劳林公式
	\section{欧拉-麦克劳林公式}
		我们直观上会认为,一个连续的级数和应该可以用这一整数区间的积分来大致模拟。一些结果也的确呈现了这个特征,比如以下有关调和级数的命题:
		
		$\sum\limits_{n=1}^{x} \frac{1}{n} = \int_{1}^{x} \frac{1}{t}\d t +o(1)$
		
		特别地,其中的余项有$o(1)\to\varepsilon$,$\varepsilon$为欧拉常数.
		
		我们对这里的余项做探讨,探究何时可以积分和求和的差值为常数阶,以及这一余项大小如何。
		\subsection{伯努利多项式}
		\begin{para}{0}
			\point{定义}
			\begin{defn}{伯努利多项式的级数定义}{}
				满足以下级数的函数族$B_n(x)$称为伯努利多项式
				\begin{equation}
					\frac{z e^{xz}}{e^z-1}=\sum\limits_{n=0}^{\infty} \frac{B_n(x)}{n!}z^n
				\end{equation}
			\end{defn}
		\end{para}
		\begin{them}{欧拉-麦克劳林公式}{}
			假设$f(x)$无穷阶可导,那么:
			\begin{equation}
				\sum\limits_{a\leq n<b}f(n) = \int_{a}^{b}f(x)\d x+\sum\limits_{k=1}^{\infty}\frac{B_k}{k!}f^{(k-1)}(x)\big|_{a}^{b}+{(-1)}^{m}\int_{a}^{b}\frac{B_m({x})}{m!}f^{(m)}(x)\d x
			\end{equation}
			
		\end{them}
	
	%附录:其他四种三角函数的泰勒展开
	\section{其他四种三角函数的泰勒/洛朗展开}
		\subsection{$\cot x$}
			借助伯努利数或者Gamma函数,分别可以将余切分解为两种形式
			\begin{them}{余切函数的洛朗展开}{}
				\begin{equation}
					\cot x = \sum\limits_{n=0}^{\infty}\frac{2(-1)^n B_{2n} (2x)^{2n-1}}{(2n)!}
				\end{equation}
				这一展开是$\cot x$的洛朗展开而非泰勒展开,因为$\cot x \sim \frac{1}{x},x\to 0$,在0处不解析
			\end{them}
			\begin{proof}
				依余切函数定义:
				$\cot x = \frac{i(e^{ix}+e^{-ix})}{e^{ix}-e^{-ix}}$
				
				$=i\frac{e^{2ix}+1}{e^{2ix}-1}$
				
				$=i\left(1+\frac{2}{e^{2ix}-1}\right)$
				再利用伯努利数的级数定义:
				
				$\cot x = i+2i\cdot \frac{1}{2ix} \sum\limits_{n=0}^{\infty} \frac{B_n (2ix)^n}{n!}$
				
				$\cot x = i+2i\sum\limits_{n=0}^{\infty} \frac{B_n (2ix)^{n-1}}{n!}$
				因为伯努利数的奇数项只有$B_1=-\frac{1}{2}$,其余全为0,故:
				
				$\cot x = i+2i\left(\sum_{n=0}^{\infty} \frac{B_2n (2ix)^2n-1}{2n!} -\frac{1}{2}\right)$
				
				$=\sum\limits_{n=0}^{\infty} \frac{2 B_{2n} (i)^{2n} (2x)^{2n-1}}{2n!}$
				$=\sum\limits_{n=0}^{\infty}\frac{2(-1)^n B_{2n} (2x)^{2n-1}}{(2n)!}$
			\end{proof}
			\begin{them}{余切函数的第二种级数展开}{}
				\begin{equation}
					\cot z = \frac{1}{z}+\sum\limits_{n=0}^{\infty} \left(\frac{1}{z-n\pi}+\frac{1}{z+n\pi}\right)
				\end{equation}
			\end{them}
			\begin{proof}
				利用正弦函数的无穷乘积展开式:
				
				$\sin \pi z = \pi z \prod\limits_{n=1}^{\infty} \left(1-\frac{z^2}{n^2}\right)$
				
				将$\pi z$移至左侧并取对数,得:
				
				$\ln \frac{\sin \pi z}{\pi z} = \sum\limits_{n=1}^{\infty} \ln \left(1-\frac{z^2}{n^2}\right)$
				
				双侧求导,得:
				
				$\frac{\pi z}{\sin \pi z}\left(\frac{\cos \pi z}{z}-\frac{\sin \pi z}{\pi z^2}\right) = \sum\limits_{n=1}^{\infty} \frac{-2z}{n^2 (1-\frac{z^2}{n^2})}$
				
				进行一些基本运算并对级数内部裂项,得:
				
				$\pi \cot \pi z = \frac{1}{z}+\sum\limits_{n=1}^{\infty}\left(\frac{1}{z-n}+\frac{1}{z+n}\right)$
				
				$\Rightarrow \cot \pi z = \frac{1}{\pi z}+\sum\limits_{n=1}^{\infty}\left(\frac{1}{\pi z-\pi n}+\frac{1}{\pi z+\pi n}\right)$
				
				$\Rightarrow \cot z = \frac{1}{z}+\sum\limits_{n=0}^{\infty} \left(\frac{1}{z-n\pi}+\frac{1}{z+n\pi}\right)$
			\end{proof}
			这一证明的注意力很强,其整体思路是:将$\cot$转换为$\cos$和$\sin$的除法,再利用正弦函数的导数,构造相应的对数除式,而恰好注意到,$\sin z$的无穷乘积展开正适合构造对数求和。
		\subsection{$\tan x$}
			\begin{them}{正切函数的Taylor展开}{}
				\begin{equation}
					\tan x = \sum\limits_{n=1}^{\infty} \frac{(-1)^{n-1}2^{2n}(2^{2n}-1)B_{2n}}{(2n)!}x^{2n-1}
				\end{equation}
			\end{them}
			\begin{proof}
				注意到:$\cot x = -2 \cot 2x +\cot x$
				
				因为:$-2\cot 2x+\cot x = -2\frac{\cos 2x}{\sin 2x}+\frac{\cos x}{\sin x}$
				
				$=-\frac{\cos^2 x-\sin^2 x}{\sin x\cos x}+\frac{\cos x}{\sin x}$
				
				$=\frac{\sin^2 x-\cos^2 x}{\sin x\cos x}+\frac{\cos^2 x}{\sin x\cos x}$
				
				$=\frac{\sin^2 x}{\sin x\cos x}$
				
				$=\frac{\sin x}{\cos x}=\cot x$
				
				$\therefore \cot x = -2\sum\limits_{n=0}^{\infty}\frac{2(-1)^n B_{2n} (4x)^{2n-1}}{(2n)!}+\sum\limits_{n=0}^{\infty}\frac{2(-1)^n B_{2n} (2x)^{2n-1}}{(2n)!} $
				
				$=-\sum\limits_{n=0}^{\infty}\frac{2(-1)^n B_{2n} (1-2^{2n}) (2x)^{2n-1}}{(2n)!}$
				
				$=\sum\limits_{n=0}^{\infty}\frac{(-1)^{n-1} B_{2n} 2^{2n}(2^{2n}-1) x^{2n-1}}{(2n)!}$
				
				又当$n=0$时一般项为0,
				
				$\therefore \cot x = \sum\limits_{n=1}^{\infty}\frac{(-1)^{n-1} 2^{2n}(2^{2n}-1) B_{2n}}{(2n)!}  x^{2n-1}$
			\end{proof}
			\begin{proof}
				怎么开始套娃了(恼)
			\end{proof}
	
	%附录:不定积分初等性判定
	\section{原函数初等性的判定方法}
		\subsection{切比雪夫定理}
			\begin{them}{切比雪夫定理}{}
				设$m,n,p\in \Q-\{0\}$,那么以下积分
				\begin{equation}
					\int x^m(a+bx^n)^p \d x
				\end{equation}
				初等的充要条件是:$p,\frac{m+1}{n},\frac{m+1}{n}+p$中至少有一个为整数
			\end{them}
		\subsection{刘维尔定理}
	
	%附录:超越积分的特殊解法
	\section{一些超越积分的特殊解法}
		\subsection{Direchlet积分}
		
	
	\ifx\allfiles\undefined
\end{document}
\fi