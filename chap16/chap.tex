\ifx\allfiles\undefined
\documentclass[12pt, a4paper, oneside, UTF8]{ctexbook}
\def\path{../config}
\input{../config/_config}
\begin{document}
	% \input{../config/cover}
	\else
	\fi
	
	\chapter{一致收敛性、函数项级数与函数族的基本运算}
	在之前章节的讨论中,曾经涉及了级数一般项是函数的级数,也就是所谓的函数项级数。
	
	在此之前,我们利用了所谓“逐点收敛”,即对每一变量取值收敛。但是,一些例子中我们发现这种收敛性并不具备很好的性质。
	
	我们提出一致收敛性这一全新的收敛性,这一性质可以允许级数仅仅需要少量条件就可以拥有微分、积分上的良好性质
	
	\section{逐点收敛性和一致收敛性}
		\subsection{逐点收敛性}
			\begin{defn}{逐点收敛性}{}
				考虑函数列$f_n : X \to \R$.如果在点$x \in X$,$\{f_n(x),n \in \}$收敛,则称$\{f_n(x),n \in \N\}$在点$x$收敛
				
				使得$\{f_n(x),n \in \N\}$收敛的点的集合称为收敛集。
				
				$\{f_n(x),n \in \N\}$在其收敛集上产生的极限$f(x) = \lim\limits_{n\to\infty}f_n(x)$称为极限函数,同时称$\{f_n(x),n \in \N\}$逐点收敛于$f(x)$
			\end{defn}
	
	\ifx\allfiles\undefined
\end{document}
\fi